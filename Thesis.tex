\documentclass[11pt]{report}
\usepackage[utf8]{inputenc}
\usepackage{graphicx}
\usepackage[margin=1in]{geometry}
\usepackage{placeins}
\usepackage{glossaries}
\graphicspath{ {images/} }

\title{
{ Procedural Plant Generation and Simulated Plant Growth }\\
{\large Massey University}
\\
\vspace{2cm}
{\includegraphics[scale=0.35]{titlepage.png}}
\vspace{2cm}
}


\author{Matthew Crankshaw}
\date{25 February 2019}


\makeglossaries
\newglossaryentry{C/C++}
{
	name=C/C++, 
	description={Refers to the C and C++ programming languages}
}

\newglossaryentry{OpenGL}
{
	name=OpenGL, 
	description={The Open Graphics Library is a cross-platform, cross-language application programming interface used in creating graphics applications}
}

\newacronym{api}{API}{Application Programming Interface}
\newacronym{glm}{GLM}{OpenGL Mathematics Library}
\newacronym{glsl}{GLSL}{OpenGL Shading Language}
\newacronym{stl}{STL}{Standard Template Library}
\newacronym{glfw}{GLFW}{Graphics Library Framework} 

\begin{document}

\maketitle

\chapter*{Acknowledgements}

\chapter*{Abstract}

\tableofcontents
\listoffigures


\chapter{Introduction}

Here I will introduce the project and the thesis in general.

\begin{flushleft}

Procedurally generating three dimensional models of plant-life is a difficult task, largely due to the complex branching structures and how different species of plants can vary so wildly in structure and detail. It has only been in the last few decades, in the boom of computer graphics that research has been conducted into how to best generate three dimensional models of plant-life. There are three main techniques for generating plant models, these include extracting and reconstructing real world data, using modeling software or using a procedural or rule based generator. This thesis explores the procedural generation and simulation of three dimensional trees using a Lindenmayer System (L-system), as well as exploring the advantages and disadvantages of different types of L-systems.

\end{flushleft}

\section{Introduction to Procedural Generation}

\begin{flushleft}

Procedural generation is used in many different areas and applications particularly in computer graphics, particularly when generating naturally occuring structures such as plants or terrain. There is an aspect of randomness that is implicit when talking about procedural generation. Generated structures should fit a certain description that is given to it, but it must also provide some form of variation such that two instances of the same description will almost never be identical. There are three main methods for procedurally generating models of trees, these are genetic algorithms \cite{haubenwallner2017shapegenetics}, space colonisation algorithms\cite{juuso2017procedural} and the Lindenmayer system. Both genetic algorithm and space colonisation algorithms are similar in that they require a description in 3D space of which describes what the trees general size and shape should look like, the algorithm is then responsible for creating branches and matching it to that original template. L-systems on the other hand operate quite differently, the L-system defines a formal grammar, which contains a set of sybols that belong to an alphabet. A set of symbols is then chosen as the starting point and a set of production rules is used to dictate which symbols can be rewriten and what they will be rewritten with. A full explanation of L-system will be spoken about in chapter \label{l-system chapter}. In essence, the L-system uses the set of production rules to generate a structure that follows those rules. This requires specifying a template or description of a plant or tree, that describes the necessary information to generate the overall structure and data algorithmically. This structure and data can then be used to render a model within a three dimensional application. Trees can have complex and random structures, however, with closer observation, trees of a similar species have very obvious traits and features, for instance a palm tree (Arecaceae) has long stright trunks with leaves exclusively near the top, the leaves are long, compound leaves, branching in all different directions. Comparatively a pine tree has a long staight trunk with many branches coming off in different directions pupendicular to the ground, from its base to the top of the trunk. These are two very different species of trees and look quite different, however they share very similar properties. The challenge behind procedurally generating and simulating trees is how to provide a human readable grammar that describes in sufficient detail, how it should generate and render the three dimensional model, whilst allowing for randomness and variety within the generation process. It must be relatively straightforward and intuitive to define the procedural generation description, and must accuratly represents it, furthermore, the description must be able to fit many different species of trees with varying charactoristics, and must not be limited to only known species of trees, as some graphics applications may require something that is other-wordly.

\end{flushleft}

\section{Introduction to Rewriting Systems}

\begin{flushleft}

String rewriting also known as rewrite systems are the fundimental concept behind L-systems. In their most basic form rewrite systems are a set of symbols or states and a set of relations that dictate how they transform from one state to the other. Rewrite systems can be non-deterministic, meaning that there could be a transition that depends on a condition being met or neighbouring states to be of a certain type. Using this rewriting concept any preceeding state can rely upon the current state as well as any conditions neccessary for transformation, otherwise the state will remain the same until such a point that the condition is satisfied. 

\end{flushleft}

\section{Motivations}
 
\begin{flushleft}
One of the most time consuming parts for digital artists and animators is creating differing variations of the same basic piece of artwork. In most games and other graphics applications environment assets such as trees, plants, grass, algae and other types of plant life make up the large majority of the assets within a game. Creating a tree asset can take a skilled digital artist more than an hour of work by hand, The artist will then have to create many variations of the same asset in order to obtain enough variation that a user of that graphics application would not notice that the asset has been duplicated. If you multiply this by the number of assets that a given artist will have to create and then modify, you are looking at an incredible number of hours that could potentially be put to use creating much more intricate and important assets.\\
\vspace{5mm}
In addition to the huge number of development hours required, it is also important to note that graphics assets are then stored in large data files, describing the geometry and textures and other information. If we require three very similar plants, we have to store three separate sets of data. Procedurally generating plants can avoid this wasteful data storage entirely. We could just store one specification or description of set of similar plants we would like to create, then procedurally generate the geometry during the running of the program.\\
\end{flushleft}

\section{Research Aims and Objectives}

\begin{flushleft}

To develop upon the Lindenmayer System in order to procedurally generating the structure of plant life in real time, in a way that allows us to specify the species, or overall look of the plant as well as introduce variation in order to produce plants that look similar, but can vary in shape, size and branching structure. \\

\vspace{5mm}

I will also investigate using the Lindenmayer System to specify aspects of the plant life that enables the simulation of physical behaviour such as external forces like gravity and wind, and thus having a specification not only for the development of the plants structure but also of its physical behaviour. \\

\vspace{5mm}

Finally, I will be investigating a method of generating a 3D mesh for the given structure of plant, where the branches are seamlessly connected together, and textures are intelligently mapped onto the generated 3D mesh.\\ 

\end{flushleft} 


\section{Scope and Limitations}

\begin{flushleft}

For the purpose of this thesis, we will only be focusing on larger plant life, such as flowers, bushes and trees. We will not be focusing on algae or fungi as these types of plants are usually better represented with specialised texturing in modern 3D applications.\\

\vspace{5mm}



\end{flushleft}

\section{Timeframe}

\begin{flushleft}

This research will be carried out over the period of a full year, from the 20\textsuperscript{th} of February 2019 through to the 20\textsuperscript{th} of February 2020. 

\end{flushleft}

\section{Structure of Thesis}






\chapter{Outline}

\begin{flushleft}

Aristid Lindenmayer is a well-known biologist who started work on what would become known as the Lindenmayer System or L-system for short. Lindenmayer initially intended the L-systems to be to be used to describe the development of simple organisms such as algae and bacteria. More recently the concept has been adapted to be used to describe larger organisms such as plants and trees. L-systems have also been used to describe non organic structures like music. \cite{worth2005growing} \\

\vspace{5mm}

An L-system at its core is a formal grammar made up of an \textit{alphabet} of symbols which are put together into strings, a set of rules is used to determine whether a symbol in the string should be rewritten with another symbol or string. What we end up with is a string of symbols which we can refer to as a set of states, for each state the rules determine what symbols to rewrite and what they should be replaced with or if they should be replaced at all.\\

\vspace{5mm}


In section \ref{Simple DOL-systems} below, I will be going into detail about a simple type of L-system called a Deterministic 0L-system.  D0L-systems serve as a good way to introduce the concept of an L-system.

\end{flushleft}

\section{Simple DOL-system} \label{Simple DOL-systems}

\begin{flushleft}

According to Prusinkiewicz and Hanan a simple type of L-systems is known as a deterministic 0L systems, where the string refers to the sequence of cellular states and the term '0L system' abbreviates 'Lindenmayer system with zero-sided interactions'.  With D0L systems there are only three major parts. There is a set of symbols known as the (\textit{alphabet}), the starting string or (\textit{axiom}) and the state transition rules (\textit{rules}). The alphabet is a set of states. The starting string or \textit{axiom} is the starting point containing one or more states. The transition rules dictate whether a state should remain the same or transition into a different state, remain the same or even disappear. \cite{prusinkiewicz2013lindenmayer}. \\

\vspace{5mm}

Below is an example of a deterministic 0L system: \\

\vspace{5mm}

We are given the \textit{alphabet} with symbols: A, B \\ 
The \textit{axiom}: A \\
The \textit{rule} set: \\ 
A $\rightarrow$ AB \\
B $\rightarrow$ A \\

\vspace{5mm}

The symbol $\rightarrow$ can be verbalised as "replaced by". Therefore, the first rule is said to be, string 'A' is replaced by string 'AB' and the second rule is said to be 'B' is replaced by the string 'A'.\\
To start we take the first state in the \textit{axiom} which, in this case is the symbol 'A', we then check it against the first rule which is 'A', if the current state matches the rule state we replace 'A' with whatever the rules successor is, which is 'AB'. We would then move onto the next state in the axiom, however there is only one state in the axiom, 'A' so we are finished with the first generation. The states 'AB' then becomes the new starting string for the first generation. We can then continue by matching the rules once again to the new starting string. Below I have shown the string for each generation up to the sixth generation.\\

\vspace{5mm}

0.) A \\
1.) AB \\
2.) ABA \\
3.) ABAAB \\
4.) ABAABABA \\
5.) ABAABABAABAAB \\

\vspace{5mm}

This rewriting of strings using a set of rules is ultimately the underlying concept behind L-systems. There are several improvements that can be made to this type of L-system in order to accommodate for more complex and intricate structures. I will be talking about these in more detail in the following sections, however some important improvements are: constants, variables, branching constructs, parametric l-systems, conditional rules and random values. //

\vspace{5mm}

An example of how an L-system can represent a real-life biological structure would be Prusinkiewicz and Lindenmayer's simulation of a blue-green bacteria known as \textit{Anabaena catenula}\\

\vspace{5mm}

Prusinkiewicz and Lindenmayer created the following DOL-system representation shown below in the following grammar: \\

\vspace{5mm}

$w$ : $ a\textsubscript{r} $\\
\textit{p1} : $ a\textsubscript{r} $ $\rightarrow$ $a\textsubscript{l}b\textsubscript{r}$ \\
\textit{p2} : $ a\textsubscript{l} $ $\rightarrow$ $b\textsubscript{l}a\textsubscript{r}$ \\
\textit{p3} : $ b\textsubscript{r} $ $\rightarrow$ $a\textsubscript{r}$ \\
\textit{p4} : $ b\textsubscript{l} $ $\rightarrow$ $a\textsubscript{l}$ \\

\vspace{5mm}

The value $w$ is there to specify the axiom which is this case has the value of $ a\textsubscript{r} $. \textit{p1}, \textit{p2}, \textit{p3}, \textit{p4} are the names of the rules that follow the semi-colon. In order to simulate Anabaena catenula we need four rules. \\
According to Prusinkiewicz and Lindenmayer "Under a microscope, the filaments appear as a sequence of cylinders of various lengths, with $a$-type cells longer than $b$-type cells. And the subscript $l$ and $r$ indicate cell polarity, specifying the positions in which daughter cells of type $a$ and $b$ will be produced. \cite{prusinkiewicz2012algorithmic} \\

\vspace{5mm}

The first five generations can be written as follows: \\

\vspace{5mm}

0.) $a_r$ \\
1.) $a_l b_r$ \\
2.) $b_l a_r a_r$ \\
3.) $a_l a_l b_r a_l b_r$ \\
4.) $b_l a_r b_l a_r a_r b_l a_r a_r$ \\
5.) $a_l a_l b_r a_l a_l b_r a_l b_r a_l a_l b_r a_l b_r$ \\

\vspace{5mm}



\end{flushleft}

\section{Constants and Variables} \label{constants variables}

\begin{flushleft}

Constants are symbols or states which don't have any significant value during the rewriting process and therefore will remain the same between generations however they do have significance when the final string is being interpreted and furthermore represented. There are a number of constants that have a fixed meaning when interpreted and are therefore known as commands. These values are:

\vspace{5mm}

$\bullet$ F: 				\hspace{10mm}  		Move forward by a specified distance whilst drawing a line \\
$\bullet$ f: 				\hspace{10mm} 		Move forward by a specified distance without drawing a line \\
$\bullet$ +: 				\hspace{10mm} 		Yaw to the right specified angle. \\
$\bullet$ -: 				\hspace{10mm} 		Yaw to the left by a specified angle.  \\
$\bullet$ /: 				\hspace{10mm} 		Pitch up by specified angle. \\
$\bullet$ $\backslash$: 	\hspace{10mm} 		Pitch down by a specified angle.  \\
$\bullet$ $\hat{}$: 		\hspace{10mm} 		Roll to the right specified angle. \\
$\bullet$ \&:				\hspace{10mm}  		Roll to the left by a specified angle.  \\

\vspace{5mm}

The question then remains why they should be interpreted using these commands and how are the instructions interpreted? As with any grammar, there are numerous ways of interpreting it. One method proposed by Przemyslaw Prusinkiewics is "To generate a string of symbols using an L-system, and to interpret this string as a sequence of commands which control a 'turtle'" \cite{prusinkiewicz1986graphical}.\\

\vspace{5mm}

When talking about a turtle, prusinkiewicz is referring to turtle graphics. Turtle graphics is a type of vector graphics that can be carried out with instructions. It is named a turtle after one of the main features of the Logo programming language. The simple set of turtle instructions listed below, can be displayed as figure \ref{basic turtle}\\

\vspace{5mm}

1. Move forward by 1.\\
2. Rotate right by 90 degrees.\\
3. Move forward by 1.\\
4. Rotate left by 90 degrees \\
5. Move forward by 1. \\
6. Rotate left by 90 degrees. \\
7. Move forward by 1. \\
8. Rotate right by 90 degrees. \\
9. Move forward by 1.\\

\vspace{5mm}

\begin{figure}[htbp]
	{\centering
		\vspace{7px}
		\includegraphics[scale=0.5]{Diagrams/basic_turtle.png}
		\caption{Diagram showing a turtle interpreting simple L-system string.} \label{basic turtle}
	}
\end{figure}
\FloatBarrier

\vspace{5mm}

There are a further two commands which I will be covering in detail in section \ref{branching}. We can also have constant numerical values that can be used. For instance, we could pass in a constant value of 1.0 as a parameter to the forward instruction as follows.

\vspace{5mm}

F(1.0)+F(1.0)-F(1.0)+F(1.0)

\vspace{5mm}

In doing this, we can specify that we would like to move forward by a specified amount. In this case we would like to move forward by 1.0 unit length. We will be covering parametric L-systems in detail in section \ref{parametric}.

\end{flushleft}

\section{Branching} \label{branching}

\begin{flushleft}

In the previous section there are two turtle commands in particular which were  not covered. These are the square bracket commands '[', ']'. The square bracket characters instruct the turtle object to save its position and rotation for the purpose of being able to restore that saved position and rotation later on. This allows the turtle to jump back to a previous position, facing the same direction as it was before. We can then branch off in a different direction.\\

\vspace{5mm}

A way to keep track of these saved locations, is in the form of a stack structure. Each time the '[' is called the current position and orientation of the turtle is saved to the top of the stack. While conversely when the ']' is called we restore the turtles position back to whatever position and orientation is stored on the top of the stack. \\

\vspace{5mm}

An example of this can be shown below in figure 2.2.\\

\begin{figure}[htbp]
	{\centering
		\vspace{7px}
		\includegraphics[scale=0.5]{Diagrams/branching_turtle.png}
		\caption{Diagram showing a turtle interpreting an L-system incorporating branching.}
	}
\end{figure}
\FloatBarrier

\end{flushleft}

\section{Parametric OL-system} \label{parametric}

\begin{flushleft}

With simplistic L-systems like the algae representation above, there are a number of details that are skipped over when making this simplistic representation. (talk about the representation for both parameterized and non-parameterised Algae systems). \\
When it comes to representing plants as L-systems a simplistic approach would be to just assume that the width and length of each branch section is constant and will not vary depending on where in the tree it is. We could also assume that the branching angles is also constant, say 25 degrees. The downside to this is that the width, height and branching angles would have to be hard coded into the turtle interpretation. Apart from making for a plant that is not very realistic. This really leaves us with two options, do we increase the complexity of the L-system grammar or increase the complexity of the interpretation of the resulting string.\\

\vspace{5mm}

In order to more accurately model plants, we would like to consider the branch width, branch height, branching angles and any other information we need in order to generate a complete 3D model of that L-system. We could even as far as to specify what kind of 3D models or textures to render.\\
For instance, defining an complex L-system grammar, with a less complexity in the interpreting system, we are given a huge amount of flexibility to define parameters that can accurately represent exactly how the L-system should be interpreted, furthermore the complexity is with the L-system grammar, so there is no need for you to make any changes to the interpreter itself. When compared to a simple L-system, you may have the advantage of being able to define simple a L-system grammar, but you would have to change and manipulate the way that the L-system is interpreted each time. Which is tedious and makes for an undesirable solution. \\

\vspace{5mm}

This is where parametric L-systems become very useful. Parametric L-systems allow us to define parameters for states. In section \ref{constants variables}. I showed an example F(1.0)+F(1.0)-F(1.0)+F(1.0), where 1.0 is the measure of distance that the turtle should move forward. This is a parametric L-system where the state F has a parameter of 1.0. Which can be interpreted as "move forward by a distance of 1.0".

\vspace{5mm}

A parametric l-system be represented as the following: \\

\vspace{5mm}

\hspace*{3cm} n = 8\\

\vspace{5mm}

\hspace*{3cm} R 1.456\\
\hspace*{3cm} r1 85\\
\hspace*{3cm} wr 0.707\\

\vspace{5mm}

\hspace*{3cm} w : A(5)\\

\vspace{5mm}

\hspace*{3cm} p1 : A(w) : * : F(1)!(w)[/(r1)A(w*wr)][$\backslash$(r1)A(w*wr)]\\
\hspace*{3cm} p2 : F(s) : * : F(s*R)\\

\vspace{5mm}

The above l-system gives the resulting representation shown below in figure 3.8. 

\begin{figure}[htbp]
	{\centering
		\vspace{7px}
		\includegraphics[scale=0.20]{ParametricLsystem/branchingPattern.png}
		\caption{3D Parametric L-system.}
	}
\end{figure}

Similarly, to the simple DOL-systems in section \ref{Simple DOL-systems}, n refers to the number of iterations that we would like to rewrite the L-system. w refers to the Axiom string or starting states, the "\#define R 1.456" states that we are defining a constant number that will be used somewhere in the production rules or Axiom, the constant will be referred to with the name R and will have the value 1.456. \#P1, \#P2 define the production rules P1 and P2. The parametric L-system introduces the concept of a module.\\
A module is a state or variable which has zero or more parameters. The Axiom A(5) is a module with one parameter which is the number 5. A parameter can either be a number, variable or an expression of variables and numbers. For instance A(a + 1, a * b) is a valid module with the name A with two parameters a + 1 and a * b, where a and b are variables, however this is only a valid module if the value of a and b have been defined previously. Each module is treated as a single instruction, it will either be overwritten when matched with a production rule or the expression of each parameter is evaluated and is left unchanged for use when interpreted.\\
Each production rule is made up of four parts. The name, the predecessor module, the condition and the successor modules, each part is separated by a colon. Therefore the predecessor for P1 is A(w), the condition is a '*' which means that in this case there are no conditions, and the successor is F(1)!(w)[/(r1)A(w*wr)][$\backslash$(r1)A(w*wr)]. I will cover conditions in a later section.\\

\vspace{5mm}

Initially we iterate through the Axiom modules and compare them to the production rule. A match is determined if they meet three criteria.\\

\vspace{5mm}

$\bullet$ The name of the axiom module matches the name of the production predecessor. \\
$\bullet$ The number of parameters for the axiom module is the same as the number of parameters for the production predecessor. \\
$\bullet$ The condition of the production evaluates to true. If there is no condition, then the result is true by default.\\

\end{flushleft}

\subsection{Representing Randomness}
\begin{flushleft}

Randomness is an essential part of nature. If there was no randomness in plant life, we would end up with very symetric and unrealistic plants. Randomness is also responsible for creating variation in the same L-system. A L-system essentially describes the structure and species of a plant. It describes everything from how large the trunk of the tree is, to how many leaves there are on the end of branch, or even if it has flowers or not. However if there is no capability to have randomness in the generation of the L-system then we will always end up with the exact same structure. 
\vspace{5mm}
Below is a simple example of how randomness can be used to create variation.

\end{flushleft}   

\begin{figure}[htbp]
	\raggedright
	\textbf{\underline{Random Fractal:}} \\
	\#n = 2; \\
	\#w : !(0.2)F(1.0); \\
	\#p1 : F(x) : * : F(x)[+(25)F(x)][-(25)F(x)]+(\{-20.0, 20.0\})F(x)-(\{-20.0, 20.0\})F(x);\\
	\vspace{10mm}
	{\centering
		\vspace{7px}
		\includegraphics[scale=0.20]{Diagrams/RandomTrees.png}
		\caption{Different Variations of the Same L-system with Randomness Introduced in The Angles. \label{figRandomness}}
	}
\end{figure}
\FloatBarrier

\begin{flushleft}

In figure \ref{figRandomness} there are four variations of the same L-system using randomness, We can specify that we would like to create a random number by using the expression \{-20.0, 20.0\}. The curly braces signify that what is contained is a random number range, ranging from the minimum value as the first floating point value and the maximum value as the second floating point value separated by a comma. If both values are the same for instance +(\{10.0, 10.0\}) this is equivilant to +(10.0).

\end{flushleft}

\subsection{Representing L-system Conditions}

\begin{flushleft}

For a module to match a rule in the rule set, we need to check three criteria. The name of the rule in the predecessor matches the module name. Secondly, the number of parameters in the predecessor are the same as the number of parameters in the module. Finally, we check if the condition in the rule evaluates to true. \\

\vspace{5mm}

In this section we will be talking about the third point in the criteria above, ie., The condition must be true in order for a module to match a production rule. A condition in a rule allows us to have multiple rules that are the same in terms of the module name and the number of parameters, however they require a particalar condition to be met in order to match that rule. \\
There are six different condition operators that can be used, similar to the programming languages C/C++, as listed below.

\vspace{5mm}

$\bullet~$  '$==$'  - Equal to \\
$\bullet~$  '$!=$' - Not Equal to \\
$\bullet~$  '$>$' - Greater than \\
$\bullet~$  '$<$' - Less than \\
$\bullet~$  '$>=$' - Greater or Equal to \\
$\bullet~$  '$<=$' - Less or Equal to \\

\vspace{5mm}

Take the following L-system as an example:\\

\vspace{5mm}

\#n = 2 - 4 - 6; \\
\#object F BRANCH; \\
\#object L LEAF; \\
\#object S SPHERE; \\

\vspace{5mm}

\#define r 45; \\
\#define len 0.5; \\
\#define lean 5.0; \\
\#define flowerW 1.0; \\

\vspace{5mm}

\#w : !(0.1)I(5); \\

\vspace{5mm}

\#p1 : I(x) : x $>$ 0 : F(len)-(lean)[R({0, 100})]F(len)[R({0, 100})]I(x-1);\\
\#p2 : R(x) : x $>$ 50 : -(r)/(20)!(2.0)L(2)!(0.1); \\
\#p3 : R(x) : x $<$ 50 : -(r)$\backslash$(170)!(2.0)L(2)!(0.1); \\
\#p4 : I(x) : x $<=$ 0 : F(len)!(flowerW)S(0.3); \\

\vspace{5mm}



\end{flushleft}

\begin{figure}[htbp]
\raggedright
\textbf{\underline{Random Fractal:}} \\
	{\centering
		\vspace{7px}
		\includegraphics[scale=0.13]{Diagrams/conditionalLsystem.png}
		\caption{Condition Statements Used to Simulate the Growth of a Flower. 2nd Generation on the Left, 4th Generation in the Center and 6th Generation on the Right}
	}
\end{figure}

\FloatBarrier










\chapter{Introducing Lindenmayer Systems}

Aristid Lindenmayer is well known biologist who started work on what would become known as the Lindenmayer System or L-system for short. Originally L-systems were intended to be used as a grammar for describing the development of simple organisms. However over the years they have been found to be useful in describing larger organisms and even non organic structures like music. \cite{worth2005growing} \\
\\
In order to talk about the more complex L-systems such as the 2L-system which is used to describe structures like plants and trees. I first need to talk about how L-systems are described as well as how string rewriting works. The most simple form of L-system is a non parametric DOL-system.    

Introduction to the implementation section

\section{Language and environment}

To understand what programming language and environment will be best suited for this project, I first  provide the technical requirements that will need to be met. The programming language will need to be  relatively fast when processing the rules of the L-systems and when rewriting these strings according to those rules. \\
\\
The program will also need to interpret the strings generated by the L-system rules and be able to generate a three dimensional representation of that L-system. This representation will need to be intuative and will have to allow me to examine it from different perspectives. In order to make the representation intuative, the 3D representation should be rendered in real time at multiple frames per second. And the user should be able to use a computer mouse and keyboard to move around the 3D world. \\ 
\\
Due to these demands have decided to use the C and C++ programming language. Due to its and thoroughly tested in built standard template library, I can count on it to be reliable and fast enough for the purpose of this project. It will also allow me to use other very useful libraries for 3D graphics such as \gls{OpenGL} which is also written in \GLS{C/C++}. I will be speaking in more detail about these details of these libraries in later sections. \\
\\
In order to create a window and provide the environment for writing pixels to the screen I will make use of the Graphics Library Framework (\acrshort{glfw}). Some useful mathematics functions and facilities can be found in the \gls{OpenGL} Matematics Library (\acrshort{glm}) and possibly the most important for rendering in 3D is the Open Graphics Library or \gls{OpenGL} for short. All of these libraries together will provide me with a strong foundation of tools that I can use to approach the practical aspect of this project. \\
\\
 

\subsection{C/C++ Programming Language}

The C programming language developed by Dennis Richie and Bell Labs in 1972 has been one of the most popular programming languages for a number of decades now \cite{ritchie1975c}. The C language was then extended upon by Bjarne Stroustrup in 1985 to create the C++ programming language \cite{stroustrup2000c++}. I have included both C and C++ as the programming languages that I will be using, as they are very closely related and C code can be compiled using the C++ compiler. For the most part I will be writing C++ code and making use of its object oriented features. However, their are instances when it will be more convenient to write C code or make use of a C library. 

\subsection{Standard Template Library (\acrshort{stl})}

The \acrshort{stl} in C/C++ provides a number of useful functions, data structures and algorithms that have been extensively tested for both reliability and efficiency. The most common features I will be using are strings, vectors, stacks and input and output. It is possible that in some cases it may be more efficient to use custom data structures for the most part the \acrshort{stl} functions will be more than good enough. \cite{horton2015stl} 

\subsection{Open Graphics Library (OpenGL)}

\gls{OpenGL} is a 2D and 3D graphics \acrshort{api}

talk about \acrshort{glsl}
\cite{movania2017opengl}

\subsection{OpenGL Mathematics Library ( \acrshort{glm} )}

\subsection{Graphics Library Framework(\acrshort{glfw})}

\subsection{Git Version Control}







\section{L-system Generator}

The purpose of the L-system generator is to read a file containing any information that might be necessary for the string rewriting process. This file must contain the number of times the string will be rewritten (number of generations), a starting point (axiom) and at least one production rule, it may also contain some constant variables. \\
\\
For simple L-systems, the generator need not be too complicated. The Koch Curve L-system stated below is a good example of this. \\
\\
\textbf{Angle:} 90\\
\textbf{Axiom:} F\\
\textbf{Rules:} \\
F $\rightarrow$ F+F-F+F\\
\\
Here we have a constant value of 90 degrees, the starting point of 'F' and one rule F $\rightarrow$ F+F-F+F. This type of system is very simple to rewrite computationally. \\ 
\\
\textbf{\textit{Here we describe some pseudocode}}\\
\\
When we move onto some more complicated L-systems, such as those that use parameters which have expressions with both variables and numbers. We end up with an L-system file that is quite difficult to process and rewrite. In order to compute these complex L-systems we need to first develop a formal grammar that describes how L-system files are defined. Once we have a formalization of how to define an parametric l-system we can create a system to carry out the rewriting.

\subsection{Building a Generalised L-system Grammar}

We are now able to represent complex three dimensional tree structures in the form of a L-system rule set. In a computing sense this rule set can be seen as a type of program. In the program we define the number of generations we would like to generate, the starting point (Axiom) some constant varables (\#define) and the production rules. \\
\\
\textless generations\textgreater~ ::= "\#n" "=" \textless float\textgreater~ ";" \\
\\
\textless definition\textgreater~ ::=  "\#define" \textless variable\textgreater~ \textless float\textgreater~ ";" \\
\\
\textless axiom\textgreater~ ::=  "\#w" ":" \textless moduleAx\textgreater~ ";" \\
\\
\textless moduleAx\textgreater~  ::= \textless variable\textgreater~ $|$ "$+$" $|$ "$-$" $|$ "/" $|$ "$\backslash$" $|$ "$\hat{}$" $|$ "$\&$" $|$ "!" 

\hspace{2cm} \textless variable\textgreater~ "("  \textless paramAx\textgreater~ \textless paramListAx\textgreater~ ")"

\hspace{2cm} $|$ "$+$" "("  \textless paramAx\textgreater~ \textless paramListAx\textgreater~ ")" 

\hspace{2cm} $|$ "$-$""("  \textless paramAx\textgreater~ \textless paramListAx\textgreater~ ")" 

\hspace{2cm} $|$ "/""("  \textless paramAx\textgreater~ \textless paramListAx\textgreater~ ")" 

\hspace{2cm} $|$ "$\backslash$""("  \textless paramAx\textgreater~ \textless paramListAx\textgreater~ ")" 

\hspace{2cm} $|$ "$\hat{}$ " "("  \textless paramAx\textgreater~ \textless paramListAx\textgreater~ ")" 

\hspace{2cm} $|$ "$\&$" "("  \textless paramAx\textgreater~ \textless paramListAx\textgreater~ ")" \\
\\
\textless paramAxList\textgreater~ ::=  $\in$ $|$ ":" \textless paramAx\textgreater~ \textless paramAxList\textgreater~ \\
\\
\textless paramAx\textgreater~ ::= \textless float\textgreater~ \\
\\
\textless production\textgreater~ ::=  "\#" \textless variable\textgreater~  ":" \textless module\textgreater~ ":" \textless condition\textgreater~  ":" \textless successor\textgreater~ ";"\\
\\
\textless module\textgreater~ ::=  \textless variable\textgreater~ $|$ "$+$" $|$ "$-$" $|$ "/" $|$ "$\backslash$" $|$ "$\hat{}$" $|$ "$\&$" $|$ "!" 

\hspace{2cm} \textless variable\textgreater~ "("  \textless param\textgreater~ \textless paramList\textgreater~ ")"

\hspace{2cm} $|$ "$+$" "("  \textless param\textgreater~ \textless paramList\textgreater~ ")" 

\hspace{2cm} $|$ "$-$""("  \textless param\textgreater~ \textless paramList\textgreater~ ")" 

\hspace{2cm} $|$ "/""("  \textless param\textgreater~ \textless paramList\textgreater~ ")" 

\hspace{2cm} $|$ "$\backslash$""("  \textless param\textgreater~ \textless paramList\textgreater~ ")" 

\hspace{2cm} $|$ "$\hat{}$ " "("  \textless param\textgreater~ \textless paramList\textgreater~ ")" 

\hspace{2cm} $|$ "$\&$" "("  \textless param\textgreater~ \textless paramList\textgreater~ ")" \\
\\
\textless paramList\textgreater~ ::=  $\in$ $|$ ":" \textless param\textgreater~ \textless paramList\textgreater~ \\
\\
\textless param\textgreater~ ::= \textless float\textgreater~ \\
\\
\textless expression\textgreater~ ::=  \textless expression\textgreater~ \textless symbol\textgreater~ \textless expression\textgreater~ $|$ 
\\
\textless float\textgreater~ ::= [0-9]+.[0-9]+ \\
\\
\textless variable\textgreater~ ::= [a-zA-Z\_][a-zA-Z0-9\_]* \\

\subsection{The L-system Compiler}

\subsection{Flex Lexical Analyser} 

\subsection{Bison Parser Generator}










\section{L-system Interpreter}

\subsection{Basic 2D L-systems} 

There are a number of fractal geometry that have become well known particularly with regards to how they can seemingly imitate nature \cite{mandelbrot1982fractal}. Particularly with the geometry such as the Koch snowflake which can be represented using the following L-system.

\begin{figure}[htbp]
	\raggedright
	\textbf{\underline{Koch Curve:}} \\
	\textbf{Alphabet:} F \\
	\textbf{Constants:} +, - \\
	\textbf{Axiom:} F \\
	\textbf{Angle:} 90$^\circ$ \\
	\textbf{Rules:} \\
	F $\rightarrow$ F+F--F+F\\
	{\centering
		\vspace{7px}
		\includegraphics[scale=0.8]{KochCurve/KochCurve04.png}
		\caption{Koch Curve.}
	}
\end{figure}
\begin{figure}[htbp]
	\raggedright
	\textbf{\underline{Sierpinski Triangle:}} \\
	\textbf{Alphabet:} A, B \\
	\textbf{Constants:} +, - \\
	\textbf{Axiom:} A \\
	\textbf{Angle:} 60$^\circ$ \\
	\textbf{Rules:} \\
	A $\rightarrow$  B-A-B \\
	B $\rightarrow$ A+B+A\\
	{\centering
		\vspace{7px}
		\includegraphics[scale=0.17]{SierpinskiTriangle/SierpinskiTriangle06.png}
		\caption{Sierpinski Triangle.}
	}
\end{figure}
\begin{figure}[htbp]
	\raggedright
	\textbf{\underline{Dragon Curve:}} \\
	\textbf{Alphabet:} F, X, Y \\
	\textbf{Constants:} +, - \\
	\textbf{Axiom:} FX \\
	\textbf{Angle:} 90$^\circ$ \\
	\textbf{Rules:} \\
	X $\rightarrow$ X+YF+ \\
	Y $\rightarrow$ -FX-Y\\
	{\centering
		\vspace{7px}
		\includegraphics[scale=0.17]{DragonCurve/DragonCurve10.png}
		\caption{Dragon Curve.}
	}
\end{figure}
\begin{figure}[htbp]
	\raggedright
	\textbf{\underline{Fractal Plant:}} \\
	\textbf{Alphabet:} X, F\\
	\textbf{Constants:} +, -, [, ] \\
	\textbf{Axiom:} X \\
	\textbf{Angle:} 25$^\circ$ \\
	\textbf{Rules:} \\
	X $\rightarrow$ F-[[X]+X]+F[+FX]-X\\
	F $\rightarrow$ FF \\
	{\centering
		\vspace{7px}
		\includegraphics[scale=0.15]{FractalPlant/FractalPlant05.png}
		\caption{Fractal Plant.}
	}
\end{figure}
\begin{figure}[htbp]
	\raggedright
	\textbf{\underline{Fractal Bush:}} \\
	\textbf{Alphabet:} F\\
	\textbf{Constants:} +, -, [, ] \\
	\textbf{Axiom:} F \\
	\textbf{Angle:} 25$^\circ$ \\
	\textbf{Rules:} \\
	F $\rightarrow$ FF+[+F-F-F]-[-F+F+F]\\
	{\centering
		\vspace{7px}
		\includegraphics[scale=0.15]{FractalBush/FractalBush06.png}
		\caption{Fractal Bush.}
	}
\end{figure}

\FloatBarrier
\newpage

\subsection{The Use of L-systems in 3D applications}

L-systems have been talked about and researched since its inception in 1968 by Aristid Lindenmayer. Over the years it's usefulness in modelling different types of plant life has been very clear, however its presence has been quite absent from any mainstream game engines for the most part, these engines relying either on digital artists skill to develop individual plants or on 3rd party software such as SpeedTree. These types of software use a multitude of different techniques however their methods are heavily rooted in Lindenmayer Systems. 



\chapter{Implementation}

Introduction to the implementation section



\lettrine[lines=3]{T}{}he motion of plants is an important factor when looking to create realistic looking plant-life. It has been a topic of discussion and research for many years now, particularly with regards to grass, bushes and trees within video games. The movement is usually very subtle, but if it is completely missing a scene can start looking very unnatural, making the user feel uncomfortable. This chapter will discuss a method of simulating the physical motion of plant-life layed out by Barron et al \cite{barron2001real}. This method will be built into the parametric L-system itself in such a way that the L-system can provide the physical parameters for the simulation. This will allow a physics simulation to be run on any plant generated by the L-system.

The main technique discussed by Barron et al for simulating the motion of a system like a tree or plant, is taken from that of a particle system first described by Reeves \cite{reeves1983particle}. Particle systems can be applied to simulate phenomena like clouds, smoke, water and fire. The main advantage of particle systems is that their motion can be updated simaltaniously. This technique can be applied to the L-system representation of plant-life, where branches are split into segments making up a skeleton of joints. Each joint can represent a \say{particle} within the system, which has a dependency on all parent branches.

\section{Motion Equations}

Torque\\
$ \tau = I\alpha $ \\
Where $\tau$ is the torque, $I$ is the moment of inertia and $\alpha$ is the angular acceleration.\\
\\
$ \tau = f \otimes R $ \\
Where $\tau$ is the torque, $f$ is the force acting on the end of the branch and $R$ is the vector representing the length and orientation of the branch. \\
\\
Mass \\
$ M = \Pi~ r ^ 2 h $  \\
Where $M$ is the mass represented in kg, $r$ is the radius of the branch in meters and $h$ is the height of the branch in meters. \\
\\
Inertia\\
$ I = \frac{1}{3} M L ^ 2 $ \\ 
Where $I$ is the inertia of the branch, $M$ is the Mass of the branch and $h$ is the height of the branch. \\
\\
Angular Acceleration\\
$ \omega = \omega _0 \alpha t $ \\
Where $\omega$ is the angular velocity, $\omega _0 $ is the previous angular velocity, $\alpha$ is the angular acceleration and $t$ is the change in time. \\
\\
Next angle equation\\
$ \theta = \theta _0 + \omega _0 t + \frac{1}{2} \alpha t ^2 $ \\
Where $\theta$ is the angle, $\theta _0$ is the previous angle, $w _0$ is the previous angular velocity, t is the change in time and $\alpha$ is the angular acceleration. \\
\\

\section{Hook's Law}

$ f = -k _s x + k _d v$\\
Where $f$ is the force exerted by the spring, $k _s$ is the spring constant and $x$ is the total displacement of the spring. We then would also like to add a dampening force which is the $k _d v$ part where $k _d$ is the dampening constant and $v$ is the velocity at the end of the spring.\\


\chapter{Glossary}
\clearpage
\printglossaries

\appendix
\chapter{Appendix}
\section{Appendix 1}

\section{Bibliography}
\bibliography{chapters/ref}
\bibliographystyle{apalike}






\end{document}



