Introduction to the implementation section

\section{Language and environment}

To understand what programming language and environment will be best suited for this project, I first  provide the technical requirements that will need to be met. The programming language will need to be  relatively fast when processing the rules of the L-systems and when rewriting these strings according to those rules. \\
\\
The program will also need to interpret the strings generated by the L-system rules and be able to generate a three dimensional representation of that L-system. This representation will need to be intuative and will have to allow me to examine it from different perspectives. In order to make the representation intuative, the 3D representation should be rendered in real time at multiple frames per second. And the user should be able to use a computer mouse and keyboard to move around the 3D world. \\ 
\\
Due to these demands have decided to use the C and C++ programming language. Due to its and thoroughly tested in built standard template library, I can count on it to be reliable and fast enough for the purpose of this project. It will also allow me to use other very useful libraries for 3D graphics such as \gls{OpenGL} which is also written in \GLS{C/C++}. I will be speaking in more detail about these details of these libraries in later sections. \\
\\
In order to create a window and provide the environment for writing pixels to the screen I will make use of the Graphics Library Framework (\acrshort{glfw}). Some useful mathematics functions and facilities can be found in the \gls{OpenGL} Matematics Library (\acrshort{glm}) and possibly the most important for rendering in 3D is the Open Graphics Library or \gls{OpenGL} for short. All of these libraries together will provide me with a strong foundation of tools that I can use to approach the practical aspect of this project. \\
\\
 

\subsection{C/C++ Programming Language}

The C programming language developed by Dennis Richie and Bell Labs in 1972 has been one of the most popular programming languages for a number of decades now \cite{ritchie1975c}. The C language was then extended upon by Bjarne Stroustrup in 1985 to create the C++ programming language \cite{stroustrup2000c++}. I have included both C and C++ as the programming languages that I will be using, as they are very closely related and C code can be compiled using the C++ compiler. For the most part I will be writing C++ code and making use of its object oriented features. However, their are instances when it will be more convenient to write C code or make use of a C library. 

\subsection{Standard Template Library (\acrshort{stl})}

The \acrshort{stl} in C/C++ provides a number of useful functions, data structures and algorithms that have been extensively tested for both reliability and efficiency. The most common features I will be using are strings, vectors, stacks and input and output. It is possible that in some cases it may be more efficient to use custom data structures for the most part the \acrshort{stl} functions will be more than good enough. \cite{horton2015stl} 

\subsection{Open Graphics Library (OpenGL)}

\gls{OpenGL} is a 2D and 3D graphics \acrshort{api}

talk about \acrshort{glsl}
\cite{movania2017opengl}

\subsection{OpenGL Mathematics Library ( \acrshort{glm} )}

\subsection{Graphics Library Framework(\acrshort{glfw})}

\subsection{Git Version Control}







\section{L-system Generator}

The purpose of the L-system generator is to read a file containing any information that might be necessary for the string rewriting process. This file must contain the number of times the string will be rewritten (number of generations), a starting point (axiom) and at least one production rule, it may also contain some constant variables. \\
\\
For simple L-systems, the generator need not be too complicated. The Koch Curve L-system stated below is a good example of this. \\
\\
\textbf{Angle:} 90\\
\textbf{Axiom:} F\\
\textbf{Rules:} \\
F $\rightarrow$ F+F-F+F\\
\\
Here we have a constant value of 90 degrees, the starting point of 'F' and one rule F $\rightarrow$ F+F-F+F. This type of system is very simple to rewrite computationally. \\ 
\\
\textbf{\textit{Here we describe some pseudocode}}\\
\\
When we move onto some more complicated L-systems, such as those that use parameters which have expressions with both variables and numbers. We end up with an L-system file that is quite difficult to process and rewrite. In order to compute these complex L-systems we need to first develop a formal grammar that describes how L-system files are defined. Once we have a formalization of how to define an parametric l-system we can create a system to carry out the rewriting.

\subsection{Building a Generalised L-system Grammar}

We are now able to represent complex three dimensional tree structures in the form of a L-system rule set. In a computing sense this rule set can be seen as a type of program. In the program we define the number of generations we would like to generate, the starting point (Axiom) some constant varables (\#define) and the production rules. \\
\\
\textless generations\textgreater~ ::= "\#n" "=" \textless float\textgreater~ ";" \\
\\
\textless definition\textgreater~ ::=  "\#define" \textless variable\textgreater~ \textless float\textgreater~ ";" \\
\\
\textless axiom\textgreater~ ::=  "\#w" ":" \textless moduleAx\textgreater~ ";" \\
\\
\textless moduleAx\textgreater~  ::= \textless variable\textgreater~ $|$ "$+$" $|$ "$-$" $|$ "/" $|$ "$\backslash$" $|$ "$\hat{}$" $|$ "$\&$" $|$ "!" 

\hspace{2cm} \textless variable\textgreater~ "("  \textless paramAx\textgreater~ \textless paramListAx\textgreater~ ")"

\hspace{2cm} $|$ "$+$" "("  \textless paramAx\textgreater~ \textless paramListAx\textgreater~ ")" 

\hspace{2cm} $|$ "$-$""("  \textless paramAx\textgreater~ \textless paramListAx\textgreater~ ")" 

\hspace{2cm} $|$ "/""("  \textless paramAx\textgreater~ \textless paramListAx\textgreater~ ")" 

\hspace{2cm} $|$ "$\backslash$""("  \textless paramAx\textgreater~ \textless paramListAx\textgreater~ ")" 

\hspace{2cm} $|$ "$\hat{}$ " "("  \textless paramAx\textgreater~ \textless paramListAx\textgreater~ ")" 

\hspace{2cm} $|$ "$\&$" "("  \textless paramAx\textgreater~ \textless paramListAx\textgreater~ ")" \\
\\
\textless paramAxList\textgreater~ ::=  $\in$ $|$ ":" \textless paramAx\textgreater~ \textless paramAxList\textgreater~ \\
\\
\textless paramAx\textgreater~ ::= \textless float\textgreater~ \\
\\
\textless production\textgreater~ ::=  "\#" \textless variable\textgreater~  ":" \textless module\textgreater~ ":" \textless condition\textgreater~  ":" \textless successor\textgreater~ ";"\\
\\
\textless module\textgreater~ ::=  \textless variable\textgreater~ $|$ "$+$" $|$ "$-$" $|$ "/" $|$ "$\backslash$" $|$ "$\hat{}$" $|$ "$\&$" $|$ "!" 

\hspace{2cm} \textless variable\textgreater~ "("  \textless param\textgreater~ \textless paramList\textgreater~ ")"

\hspace{2cm} $|$ "$+$" "("  \textless param\textgreater~ \textless paramList\textgreater~ ")" 

\hspace{2cm} $|$ "$-$""("  \textless param\textgreater~ \textless paramList\textgreater~ ")" 

\hspace{2cm} $|$ "/""("  \textless param\textgreater~ \textless paramList\textgreater~ ")" 

\hspace{2cm} $|$ "$\backslash$""("  \textless param\textgreater~ \textless paramList\textgreater~ ")" 

\hspace{2cm} $|$ "$\hat{}$ " "("  \textless param\textgreater~ \textless paramList\textgreater~ ")" 

\hspace{2cm} $|$ "$\&$" "("  \textless param\textgreater~ \textless paramList\textgreater~ ")" \\
\\
\textless paramList\textgreater~ ::=  $\in$ $|$ ":" \textless param\textgreater~ \textless paramList\textgreater~ \\
\\
\textless param\textgreater~ ::= \textless float\textgreater~ \\
\\
\textless expression\textgreater~ ::=  \textless expression\textgreater~ \textless symbol\textgreater~ \textless expression\textgreater~ $|$ 
\\
\textless float\textgreater~ ::= [0-9]+.[0-9]+ \\
\\
\textless variable\textgreater~ ::= [a-zA-Z\_][a-zA-Z0-9\_]* \\

\subsection{The L-system Compiler}

\subsection{Flex Lexical Analyser} 

\subsection{Bison Parser Generator}










\section{L-system Interpreter}

\subsection{Basic 2D L-systems} 

There are a number of fractal geometry that have become well known particularly with regards to how they can seemingly imitate nature \cite{mandelbrot1982fractal}. Particularly with the geometry such as the Koch snowflake which can be represented using the following L-system.

\begin{figure}[htbp]
	\raggedright
	\textbf{\underline{Koch Curve:}} \\
	\textbf{Alphabet:} F \\
	\textbf{Constants:} +, - \\
	\textbf{Axiom:} F \\
	\textbf{Angle:} 90$^\circ$ \\
	\textbf{Rules:} \\
	F $\rightarrow$ F+F--F+F\\
	{\centering
		\vspace{7px}
		\includegraphics[scale=0.8]{KochCurve/KochCurve04.png}
		\caption{Koch Curve.}
	}
\end{figure}
\begin{figure}[htbp]
	\raggedright
	\textbf{\underline{Sierpinski Triangle:}} \\
	\textbf{Alphabet:} A, B \\
	\textbf{Constants:} +, - \\
	\textbf{Axiom:} A \\
	\textbf{Angle:} 60$^\circ$ \\
	\textbf{Rules:} \\
	A $\rightarrow$  B-A-B \\
	B $\rightarrow$ A+B+A\\
	{\centering
		\vspace{7px}
		\includegraphics[scale=0.17]{SierpinskiTriangle/SierpinskiTriangle06.png}
		\caption{Sierpinski Triangle.}
	}
\end{figure}
\begin{figure}[htbp]
	\raggedright
	\textbf{\underline{Dragon Curve:}} \\
	\textbf{Alphabet:} F, X, Y \\
	\textbf{Constants:} +, - \\
	\textbf{Axiom:} FX \\
	\textbf{Angle:} 90$^\circ$ \\
	\textbf{Rules:} \\
	X $\rightarrow$ X+YF+ \\
	Y $\rightarrow$ -FX-Y\\
	{\centering
		\vspace{7px}
		\includegraphics[scale=0.17]{DragonCurve/DragonCurve10.png}
		\caption{Dragon Curve.}
	}
\end{figure}
\begin{figure}[htbp]
	\raggedright
	\textbf{\underline{Fractal Plant:}} \\
	\textbf{Alphabet:} X, F\\
	\textbf{Constants:} +, -, [, ] \\
	\textbf{Axiom:} X \\
	\textbf{Angle:} 25$^\circ$ \\
	\textbf{Rules:} \\
	X $\rightarrow$ F-[[X]+X]+F[+FX]-X\\
	F $\rightarrow$ FF \\
	{\centering
		\vspace{7px}
		\includegraphics[scale=0.15]{FractalPlant/FractalPlant05.png}
		\caption{Fractal Plant.}
	}
\end{figure}
\begin{figure}[htbp]
	\raggedright
	\textbf{\underline{Fractal Bush:}} \\
	\textbf{Alphabet:} F\\
	\textbf{Constants:} +, -, [, ] \\
	\textbf{Axiom:} F \\
	\textbf{Angle:} 25$^\circ$ \\
	\textbf{Rules:} \\
	F $\rightarrow$ FF+[+F-F-F]-[-F+F+F]\\
	{\centering
		\vspace{7px}
		\includegraphics[scale=0.15]{FractalBush/FractalBush06.png}
		\caption{Fractal Bush.}
	}
\end{figure}

\FloatBarrier
\newpage

\subsection{The Use of L-systems in 3D applications}

L-systems have been talked about and researched since its inception in 1968 by Aristid Lindenmayer. Over the years it's usefulness in modelling different types of plant life has been very clear, however its presence has been quite absent from any mainstream game engines for the most part, these engines relying either on digital artists skill to develop individual plants or on 3rd party software such as SpeedTree. These types of software use a multitude of different techniques however their methods are heavily rooted in Lindenmayer Systems. 

