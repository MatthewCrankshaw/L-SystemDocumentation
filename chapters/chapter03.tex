\section{Simple DOL-system}

According to Prusinkiewicz and Hanan a simple type of L-systems are those known as deterministic 0L systems, where the string refers to the sequence of cellular states and '0L system' abbreviating 'Lindenmayer system with zero-sided interactions'.  With 0L systems there are only three major parts. There is a set of symbols known as the (\textit{alphabet}), the starting string (\textit{Axiom}) and state transition rules (\textit{rules}). The alphabet is a set of states. The starting string is a starting point containing one or more states. The transition rules are rules that dictate whether a state should remain the same or transition into a different state or even disappear. \cite{prusinkiewicz2013lindenmayer}. \\
\\
An example of a deterministic 0L system: \\
\\
We are given the \textit{alphabet}: A, B \\ 
and the \textit{axiom}: A \\
and the \textit{rule} set: \\ 
A $\rightarrow$ AB \\
B $\rightarrow$ A \\
\\
The symbol $\rightarrow$ can be verbalised as "replaced by". Therefor it can be said that the string 'A' is replaced by string 'AB' and string 'B' is replaced by the string 'A'.\\
To start we take the \textit{axiom} which is this case is 'A' and we run through all of the states in this start string 'A' matches the rule: A $\rightarrow$ AB and is therefor replaced by 'AB'. 'AB' then becomes the new start string and we then match the rules once again. Below I have shown the resulting string up to six generations.\\
\\
If we then apply the rules to the L-system we find it creates the following generation structure. \\
\\
1.) A \\
2.) AB \\
3.) ABA \\
4.) ABAAB \\
5.) ABAABABA \\
6.) ABAABABAABAAB \\
\\
This rewriting of initial string using a set of rules is ultimately the underlying concept behind L-systems. There are a number of improvements that can be made to this type of L-system in order to accommodate for more complex and intricate structure. One of which is the inclusion of \textit{constants}. Constants can be considered any state that does not have a rule associated with it or remains the same from generation to generation and therefore holds a consistent value or meaning. These constants are used when the L-system is interpreted and thus holds a constant value during string rewriting, I will be covering this in section \ref{interpreting l-systems}. \\
\\
Prusinkiewicz and Lindenmayer simulated a blue-green bacteria known as \textit{Anabaena catenula} \cite{prusinkiewicz2012algorithmic}\\
\\
The DOL-system is described as follows: \\
\\
$w$ : $ a\textsubscript{r} $\\
\textit{p1} : $ a\textsubscript{r} $ $\rightarrow$ $a\textsubscript{l}b\textsubscript{r}$ \\
\textit{p2} : $ a\textsubscript{l} $ $\rightarrow$ $b\textsubscript{l}a\textsubscript{r}$ \\
\textit{p3} : $ b\textsubscript{r} $ $\rightarrow$ $a\textsubscript{r}$ \\
\textit{p4} : $ b\textsubscript{l} $ $\rightarrow$ $a\textsubscript{l}$ \\
\\
The value $w$ is there to specify the axiom which is this case has the value of $ a\textsubscript{r} $. \textit{p1}, \textit{p2}, \textit{p3}, \textit{p4} are the names of the rules that follow after the semi-colon. In order to simulate Anabaena catenula we need four rules. \\
According to Prusinkiewicz and Lindenmayer "Under a microscope, the filaments appear as a sequence of cylinders of various lengths, with $a$-type cells longer than $b$-type cells. And the subscript $l$ and $r$ indicate cell polarity, specifying the positions in which daughter cells of type $a$ and $b$ will be produced. \cite{prusinkiewicz2012algorithmic} \\
\\ 
This gives us a good real world demonstration of how even a simple DOL-system can represent something in the real world. 


\section{Interpreting L-systems} \label{interpreting l-systems}

Once we have generated the set of rules that allow us to create the L-system we are left with a string of characters which represent that particular L-system. As with any grammar, there is a number of ways of interpreting the string that is generated by the L-systems rules. One method proposed by Przemyslaw Prusinkiewics is "to generate a string of symbols using an L-system, and to interpret this string as a sequence of commands which control a 'turtle'". \cite{prusinkiewicz1986graphical}
\\
\\
A two dimensional L-system string may hold the following commands in the form of symbols. \\
\\
$\bullet$ F: 				\hspace{10mm} 		Move forward by a specified distance whilst drawing a line \\
$\bullet$ f: 				\hspace{10mm} 		Move forward by a specified distance without drawing a line \\
$\bullet$ +: 				\hspace{10mm} 		Rotate to the right specified angle. \\
$\bullet$ -: 				\hspace{10mm} 		Rotate to the left by a specified angle.  \\
$\bullet$ $[$: 				\hspace{10mm} 		Save the current position and angle. \\
$\bullet$ $]$: 				\hspace{10mm} 		Load a saved position and angle. \\
\\
Similarly a three dimensional L-system string may hold the following commands in the form of symbols. \\
\\ 
$\bullet$ F: 				\hspace{10mm}  		Move forward by a specified distance whilst drawing a line \\
$\bullet$ f: 				\hspace{10mm} 		Move forward by a specified distance without drawing a line \\
$\bullet$ +: 				\hspace{10mm} 		Yaw to the right specified angle. \\
$\bullet$ -: 				\hspace{10mm} 		Yaw to the left by a specified angle.  \\
$\bullet$ /: 				\hspace{10mm} 		Pitch up by specified angle. \\
$\bullet$ $\backslash$: 	\hspace{10mm} 		Pitch down by a specified angle.  \\
$\bullet$ $\hat{}$: 		\hspace{10mm} 		Roll to the right specified angle. \\
$\bullet$ \&:				\hspace{10mm}  		Roll to the left by a specified angle.  \\
$\bullet$ $[$: 				\hspace{10mm} 		Save the current position and angle. \\
$\bullet$ $]$: 				\hspace{10mm}		Load a saved position and angle. \\

\begin{figure}[htbp]
	{\centering
		\vspace{7px}
		\includegraphics[scale=0.5]{Diagrams/basic_turtle.png}
		\caption{Diagram showing a turtle interpreting simple L-system string.}
	}
\end{figure}
\FloatBarrier

\section{Branching Filaments}

Explain how branching works and how it can be used for generating the L-systems

\begin{figure}[htbp]
	{\centering
		\vspace{7px}
		\includegraphics[scale=0.5]{Diagrams/branching_turtle.png}
		\caption{Diagram showing a turtle interpreting an L-system incorporating branching.}
	}
\end{figure}
\FloatBarrier


\section{Basic 2D L-systems} 

There are a number of fractal geometry that have become well known particularly with regards to how they can seemingly imitate nature \cite{mandelbrot1982fractal}. Particularly with the geometry such as the Koch snowflake which can be represented using the following L-system.

\begin{figure}[htbp]
	\raggedright
	\textbf{\underline{Koch Curve:}} \\
	\textbf{Alphabet:} F \\
	\textbf{Constants:} +, - \\
	\textbf{Axiom:} F \\
	\textbf{Angle:} 90$^\circ$ \\
	\textbf{Rules:} \\
	F $\rightarrow$ F+F--F+F\\
	{\centering
		\vspace{7px}
		\includegraphics[scale=0.8]{KochCurve/KochCurve04.png}
		\caption{Koch Curve.}
	}
\end{figure}
\begin{figure}[htbp]
	\raggedright
	\textbf{\underline{Sierpinski Triangle:}} \\
	\textbf{Alphabet:} A, B \\
	\textbf{Constants:} +, - \\
	\textbf{Axiom:} A \\
	\textbf{Angle:} 60$^\circ$ \\
	\textbf{Rules:} \\
	A $\rightarrow$  B-A-B \\
	B $\rightarrow$ A+B+A\\
	{\centering
		\vspace{7px}
		\includegraphics[scale=0.17]{SierpinskiTriangle/SierpinskiTriangle06.png}
		\caption{Sierpinski Triangle.}
	}
\end{figure}
\begin{figure}[htbp]
	\raggedright
	\textbf{\underline{Dragon Curve:}} \\
	\textbf{Alphabet:} F, X, Y \\
	\textbf{Constants:} +, - \\
	\textbf{Axiom:} FX \\
	\textbf{Angle:} 90$^\circ$ \\
	\textbf{Rules:} \\
	X $\rightarrow$ X+YF+ \\
	Y $\rightarrow$ -FX-Y\\
	{\centering
		\vspace{7px}
		\includegraphics[scale=0.17]{DragonCurve/DragonCurve10.png}
		\caption{Dragon Curve.}
	}
\end{figure}
\begin{figure}[htbp]
	\raggedright
	\textbf{\underline{Fractal Plant:}} \\
	\textbf{Alphabet:} X, F\\
	\textbf{Constants:} +, -, [, ] \\
	\textbf{Axiom:} X \\
	\textbf{Angle:} 25$^\circ$ \\
	\textbf{Rules:} \\
	X $\rightarrow$ F-[[X]+X]+F[+FX]-X\\
	F $\rightarrow$ FF \\
	{\centering
		\vspace{7px}
		\includegraphics[scale=0.15]{FractalPlant/FractalPlant05.png}
		\caption{Fractal Plant.}
	}
\end{figure}
\begin{figure}[htbp]
	\raggedright
	\textbf{\underline{Fractal Bush:}} \\
	\textbf{Alphabet:} F\\
	\textbf{Constants:} +, -, [, ] \\
	\textbf{Axiom:} F \\
	\textbf{Angle:} 25$^\circ$ \\
	\textbf{Rules:} \\
	F $\rightarrow$ FF+[+F-F-F]-[-F+F+F]\\
	{\centering
		\vspace{7px}
		\includegraphics[scale=0.15]{FractalBush/FractalBush06.png}
		\caption{Fractal Bush.}
	}
\end{figure}

\FloatBarrier
\newpage

\section{The Use of L-systems in 3D applications}

L-systems have been talked about and researched since its inception in 1968 by Aristid Lindenmayer. Over the years it's usefulness in modelling different types of plant life has been very clear, however its presence has been quite absent from any mainstream game engines for the most part, these engines relying either on digital artists skill to develop individual plants or on 3rd party software such as SpeedTree. These types of software use a multitude of different techniques however their methods are heavily rooted in Lindenmayer Systems. 

\newpage

\section{Parametric L-system}

With simplistic L-systems like the algae representation above, there are a number of details that are skipped over when making this simplistic representation. (talk about the representation for both parameterized and non parameterised Algae systems). When it comes to representing trees as L-systems a simplistic approach would be to just assume that the width and length of each branch section is constant and will not vary depending on where in the tree it is. We can also assume that the angles at which a branch may split is also constant, say 25 degrees. 
The resulting representation of this L-system is a tree like structure, however it is not a very accurate representation of a real tree in nature. \\
In order to more accurately model trees we need to take into account the branch width, height, branching angles. There are two different approaches to solving this added complexity. One would be to increase the complexity of the L-system grammar and the other would be to increase the complexity of the interpretation of the L-system. \\
For instance defining an complex L-system grammar with a less complex interpreting system can give a huge amount of flexibility to define parameters that can accurately define exactly how the L-system should be interpreted, and because the complexity is with the L-system rewriting you also have the control of being able to change the L-system rules. \\

\subsection{Formalising Parametric L-system Grammar}



\textless generations\textgreater~ ::= "\#n" "=" \textless float\textgreater~ ";" \\
\\
\textless definition\textgreater~ ::=  "\#define" \textless variable\textgreater~ \textless float\textgreater~ ";" \\
\\
\textless axiom\textgreater~ ::=  "\#w" ":" \textless moduleAx\textgreater~ ";" \\
\textless moduleAx\textgreater~  ::= \textless variable\textgreater~ $|$ "$+$" $|$ "$-$" $|$ "/" $|$ "$\backslash$" $|$ "$\hat{}$" $|$ "$\&$" $|$ "!" 

\hspace{1cm} \textless variable\textgreater~ "("  \textless paramAx\textgreater~ \textless paramListAx\textgreater~ ")"

\hspace{1cm} $|$ "$+$" "("  \textless paramAx\textgreater~ \textless paramListAx\textgreater~ ")" 

\hspace{1cm} $|$ "$-$""("  \textless paramAx\textgreater~ \textless paramListAx\textgreater~ ")" 

\hspace{1cm} $|$ "/""("  \textless paramAx\textgreater~ \textless paramListAx\textgreater~ ")" 

\hspace{1cm} $|$ "$\backslash$""("  \textless paramAx\textgreater~ \textless paramListAx\textgreater~ ")" 

\hspace{1cm} $|$ "$\hat{}$ " "("  \textless paramAx\textgreater~ \textless paramListAx\textgreater~ ")" 

\hspace{1cm} $|$ "$\&$" "("  \textless paramAx\textgreater~ \textless paramListAx\textgreater~ ")" \\
\textless paramAxList\textgreater~ ::=  $\in$ $|$ ":" \textless paramAx\textgreater~ \textless paramAxList\textgreater~ \\
\textless paramAx\textgreater~ ::= \textless float\textgreater~ \\
\\
\textless production\textgreater~ ::=  "\#" \textless variable\textgreater~  ":" \textless module\textgreater~ ":" \textless condition\textgreater~  ":" \textless successor\textgreater~ ";"\\
\textless module\textgreater~ ::=  \textless variable\textgreater~ $|$ "$+$" $|$ "$-$" $|$ "/" $|$ "$\backslash$" $|$ "$\hat{}$" $|$ "$\&$" $|$ "!" 

\hspace{1cm} \textless variable\textgreater~ "("  \textless param\textgreater~ \textless paramList\textgreater~ ")"

\hspace{1cm} $|$ "$+$" "("  \textless param\textgreater~ \textless paramList\textgreater~ ")" 

\hspace{1cm} $|$ "$-$""("  \textless param\textgreater~ \textless paramList\textgreater~ ")" 

\hspace{1cm} $|$ "/""("  \textless param\textgreater~ \textless paramList\textgreater~ ")" 

\hspace{1cm} $|$ "$\backslash$""("  \textless param\textgreater~ \textless paramList\textgreater~ ")" 

\hspace{1cm} $|$ "$\hat{}$ " "("  \textless param\textgreater~ \textless paramList\textgreater~ ")" 

\hspace{1cm} $|$ "$\&$" "("  \textless param\textgreater~ \textless paramList\textgreater~ ")" \\
\textless paramList\textgreater~ ::=  $\in$ $|$ ":" \textless param\textgreater~ \textless paramList\textgreater~ \\
\textless param\textgreater~ ::= \textless float\textgreater~ \\
\\
\textless expression\textgreater~ ::=  \textless expression\textgreater~ \textless symbol\textgreater~ \textless expression\textgreater~ $|$ 
\\
\textless float\textgreater~ ::= [0-9]+.[0-9]+ \\
\textless variable\textgreater~ ::= [a-zA-Z\_][a-zA-Z0-9\_]* \\



