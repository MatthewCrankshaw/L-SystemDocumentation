\begin{flushleft}

Aristid Lindenmayer is a well-known biologist who started work on what would become known as the Lindenmayer System or L-system for short. Lindenmayer initially intended the L-systems to be to be used to describe the development of simple organisms such as algae and bacteria. More recently the concept has been adapted to be used to describe larger organisms such as plants and trees. L-systems have also been used to describe non organic structures like music. \cite{worth2005growing} \\

\vspace{5mm}

An L-system at its core is a formal grammar made up of an \textit{alphabet} of symbols which are put together into strings, a set of rules is used to determine whether a symbol in the string should be rewritten with another symbol or string. What we end up with is a string of symbols which we can refer to as a set of states, for each state the rules determine what symbols to rewrite and what they should be replaced with or if they should be replaced at all.\\

\vspace{5mm}


In section \ref{Simple DOL-systems} below, I will be going into detail about a simple type of L-system called a Deterministic 0L-system.  D0L-systems serve as a good way to introduce the concept of an L-system.

\end{flushleft}

\section{Simple DOL-system} \label{Simple DOL-systems}

\begin{flushleft}

According to Prusinkiewicz and Hanan a simple type of L-systems is known as a deterministic 0L systems, where the string refers to the sequence of cellular states and the term '0L system' abbreviates 'Lindenmayer system with zero-sided interactions'.  With D0L systems there are only three major parts. There is a set of symbols known as the (\textit{alphabet}), the starting string or (\textit{axiom}) and the state transition rules (\textit{rules}). The alphabet is a set of states. The starting string or \textit{axiom} is the starting point containing one or more states. The transition rules dictate whether a state should remain the same or transition into a different state, remain the same or even disappear. \cite{prusinkiewicz2013lindenmayer}. \\

\vspace{5mm}

Below is an example of a deterministic 0L system: \\

\vspace{5mm}

We are given the \textit{alphabet} with symbols: A, B \\ 
The \textit{axiom}: A \\
The \textit{rule} set: \\ 
A $\rightarrow$ AB \\
B $\rightarrow$ A \\

\vspace{5mm}

The symbol $\rightarrow$ can be verbalised as "replaced by". Therefore, the first rule is said to be, string 'A' is replaced by string 'AB' and the second rule is said to be 'B' is replaced by the string 'A'.\\
To start we take the first state in the \textit{axiom} which, in this case is the symbol 'A', we then check it against the first rule which is 'A', if the current state matches the rule state we replace 'A' with whatever the rules successor is, which is 'AB'. We would then move onto the next state in the axiom, however there is only one state in the axiom, 'A' so we are finished with the first generation. The states 'AB' then becomes the new starting string for the first generation. We can then continue by matching the rules once again to the new starting string. Below I have shown the string for each generation up to the sixth generation.\\

\vspace{5mm}

0.) A \\
1.) AB \\
2.) ABA \\
3.) ABAAB \\
4.) ABAABABA \\
5.) ABAABABAABAAB \\

\vspace{5mm}

This rewriting of strings using a set of rules is ultimately the underlying concept behind L-systems. There are several improvements that can be made to this type of L-system in order to accommodate for more complex and intricate structures. I will be talking about these in more detail in the following sections, however some important improvements are: constants, variables, branching constructs, parametric l-systems, conditional rules and random values. //

\vspace{5mm}

An example of how an L-system can represent a real-life biological structure would be Prusinkiewicz and Lindenmayer's simulation of a blue-green bacteria known as \textit{Anabaena catenula}\\

\vspace{5mm}

Prusinkiewicz and Lindenmayer created the following DOL-system representation shown below in the following grammar: \\

\vspace{5mm}

$w$ : $ a\textsubscript{r} $\\
\textit{p1} : $ a\textsubscript{r} $ $\rightarrow$ $a\textsubscript{l}b\textsubscript{r}$ \\
\textit{p2} : $ a\textsubscript{l} $ $\rightarrow$ $b\textsubscript{l}a\textsubscript{r}$ \\
\textit{p3} : $ b\textsubscript{r} $ $\rightarrow$ $a\textsubscript{r}$ \\
\textit{p4} : $ b\textsubscript{l} $ $\rightarrow$ $a\textsubscript{l}$ \\

\vspace{5mm}

The value $w$ is there to specify the axiom which is this case has the value of $ a\textsubscript{r} $. \textit{p1}, \textit{p2}, \textit{p3}, \textit{p4} are the names of the rules that follow the semi-colon. In order to simulate Anabaena catenula we need four rules. \\
According to Prusinkiewicz and Lindenmayer "Under a microscope, the filaments appear as a sequence of cylinders of various lengths, with $a$-type cells longer than $b$-type cells. And the subscript $l$ and $r$ indicate cell polarity, specifying the positions in which daughter cells of type $a$ and $b$ will be produced. \cite{prusinkiewicz2012algorithmic} \\

\vspace{5mm}

The first five generations can be written as follows: \\

\vspace{5mm}

0.) $a_r$ \\
1.) $a_l b_r$ \\
2.) $b_l a_r a_r$ \\
3.) $a_l a_l b_r a_l b_r$ \\
4.) $b_l a_r b_l a_r a_r b_l a_r a_r$ \\
5.) $a_l a_l b_r a_l a_l b_r a_l b_r a_l a_l b_r a_l b_r$ \\

\vspace{5mm}



\end{flushleft}

\section{Turtle Instructions} \label{turtle instructions}

\begin{flushleft}

Turtle instructions are symbols or states which will remain the same during the rewriting process, and have significance when the final resultant string is being interpret. In simple 0L-systems these instructions do not have any influence during the rewriting process, however, in parametric L-systems covered in section \ref{parametric} turtle instructions have extra functionality built in which allow then to be modified during the rewriting process. \\
There are a number of instructions that have a fixed meaning when interpreted, these values are:

\vspace{5mm}

$\bullet$ F: 				\hspace{10mm}  		Move forward by a specified distance whilst drawing a line \\
$\bullet$ f: 				\hspace{10mm} 		Move forward by a specified distance without drawing a line \\
$\bullet$ +: 				\hspace{10mm} 		Yaw to the right specified angle. \\
$\bullet$ -: 				\hspace{10mm} 		Yaw to the left by a specified angle.  \\
$\bullet$ /: 				\hspace{10mm} 		Pitch up by specified angle. \\
$\bullet$ $\backslash$: 	\hspace{10mm} 		Pitch down by a specified angle.  \\
$\bullet$ $\hat{}$: 		\hspace{10mm} 		Roll to the right specified angle. \\
$\bullet$ \&:				\hspace{10mm}  		Roll to the left by a specified angle.  \\

\vspace{5mm}

The question then remains, how are the instructions interpreted? As with any grammar, there are numerous ways of interpreting it. One method proposed by Przemyslaw Prusinkiewics is "To generate a string of symbols using an L-system, and to interpret this string as a sequence of commands which control a 'turtle'" \cite{prusinkiewicz1986graphical}. When talking about a turtle, prusinkiewicz is referring to turtle graphics. Turtle graphics is a type of vector graphics that can be carried out with instructions. It is named a turtle after one of the main features of the Logo programming language. The simple set of turtle instructions listed below, can be displayed as figure \ref{basic turtle}\\

\vspace{5mm}

Instruction 1. Move forward by 1.\\
Instruction 2. Rotate right by 90 degrees.\\
Instruction 3. Move forward by 1.\\
Instruction 4. Rotate left by 90 degrees \\
Instruction 5. Move forward by 1. \\
Instruction 6. Rotate left by 90 degrees. \\
Instruction 7. Move forward by 1. \\
Instruction 8. Rotate right by 90 degrees. \\
Instruction 9. Move forward by 1.\\

\vspace{5mm}

\begin{figure}[htbp]
	{\centering
		\setlength{\fboxrule}{1pt}
		\vspace{7px}
		\fbox{
			\includegraphics[scale=0.5]{Diagrams/basic_turtle.png}
		}
		\caption{Diagram showing a turtle interpreting simple L-system string.} \label{basic turtle}
	}
\end{figure}
\FloatBarrier

\vspace{5mm}

There are a further two commands which I will be covering in detail in section \ref{branching}. We can also have constant numerical values that can be used. For instance, we could pass in a constant value of 1.0 as a parameter to the forward instruction as follows.

\vspace{5mm}

F(1.0)+F(1.0)-F(1.0)+F(1.0)

\vspace{5mm}

In doing this, we can specify that we would like to move forward by a specified amount. In this case we would like to move forward by 1.0 unit length. We will be covering parametric L-systems in detail in section \ref{parametric}.

\end{flushleft}

\section{Branching} \label{branching}

\begin{flushleft}

In the previous section there are two turtle commands in particular which were  not covered. These are the square bracket commands '[', ']'. The square bracket characters instruct the turtle object to save its position and rotation for the purpose of being able to restore that saved position and rotation later on. This allows the turtle to jump back to a previous position, facing the same direction as it was before. We can then branch off in a different direction.\\

\vspace{5mm}

A way to keep track of these saved locations, is in the form of a stack structure. Each time the '[' is called the current position and orientation of the turtle is saved to the top of the stack. While conversely when the ']' is called we restore the turtles position back to whatever position and orientation is stored on the top of the stack. \\

\vspace{5mm}

An example of this can be shown below in figure 2.2.\\

\begin{figure}[htbp]
	{\centering
		\setlength{\fboxrule}{1pt}
		\vspace{7px}
		\fbox{
			\includegraphics[scale=0.35]{Diagrams/branching_turtle.png}
		}
		\caption{Diagram showing a turtle interpreting an L-system incorporating branching.}
	}
\end{figure}
\FloatBarrier

\end{flushleft}

\section{Parametric OL-system} \label{parametric}

\begin{flushleft}

Simplistic L-systems like the algae representation in section \ref{Simple DOL-systems} above, give us enough information to create a very basic structure of plant life, there are many details that are not included which a simple OL-system will not be able to represent. With the simplistic approach we have assumed that the width and length and branching angles of each section is constant or predefined. The result of this was that all of the details such as width and length of branchesis left up to the interpretation of the resultant L-system string. This begs the question as to how we should accurately interpret the L-system string when we are not provided the details by the L-system. The answer lies in parametric 0L-systems.

\vspace{5mm}

In this section I will outline the definition and major concepts of the parametric L-system formulated by Prusinkiewicz and Hanan in 1990 \cite{prusinkiewicz1990visualization}, and developed upon in 2012 by Prusinkiewicz and Lindenmayer \cite{prusinkiewicz2012algorithmic}. I will also be talking about some of the changes that I have made, and explaining why these changes are necessary for the purpose of this thesis.

\end{flushleft}

\subsection{Definition of a Parametric 0L-system}

\begin{flushleft}

Prusinkiewicz and Hanan define the parametric 0L-systems as a system of parametric words, where a string of letters make up a module name $A$, each module has a number of parameters associated with it. The module names belong an alphabet $V$, therefore, $A~ \in~ V$, and the parameters belong to a set of real numbers $\Re$. If $(a_1,~ a_2,~ ...,~ a_n)~ \in~ R$ are parameters of $A$, the module can be stated as $A(a_1,~ a_2,~ ...,~ a_n)$. Each module is an element of the set of modules $M~ =~ V~ \times~ \Re^*$. $\Re^*$ represents the set of all finite sequences of parameters, including the case where there are no parameters. We can then infer that $M^*~ =~ (V~ \times~ \Re^*)^*$ where $M^*$ is the set of all finite modules. \\
Each parameter of a given module corresponds to a formal definition of that parameter defined within the L-system productions. Let the formal definition of a parameter be $\Sigma$. $ E(\Sigma) $ can be said to be an arithmetic expression of a given parameter.\\ Similar to the arithmetic expressions in the programming languages C/C++, we can make use of the arithmetic operators $ +,~ -,~ *,~ \,~ \wedge{}$. Furthermore, we can have the relational expression $C(\Sigma)$, with a set of relational operators. In the literature by Prusinkiewicz and Hanan the set of relational operators is said to be $<,~ >,~ =$, I have extended this to include the relational operators $>,~ <,~ >=,~ <=,~ ==,~ !=$. Where $==$ is the 'equal to' operator and $!=$ is the 'not equal' operator, and the symbols $>=$ and $<=$ are 'greater than or equal to' and 'less than or equal to' respectively. The parentheses () are also used in order to specify precedence within an expression. The arithmetic expressions can be evaluated and will result in the real number parameter $\Re $, and the relational expressions can be evaluated to either true or false. \\

\vspace{5mm}

The parametric 0L-system can be shown as follows as per Prusinkiewicz and Hanan's definition:\\

\vspace{5mm}

\begin{equation}
G~ = (V, \Sigma, \omega, P)
\end{equation}
\vspace{5mm}

$G$ is an ordererd quadruplet that describes the parametric OL-system. $V$ is the alphabet of characters for the system. $\Sigma$ is the set of formal parameters for the system. $\omega~ \in~ (V~ \times \Re^*)^+$ is a non-empty parametric word called the axiom. Finally $P$ is a finite set of production rules which can be fully defined as:

\vspace{5mm}

\begin{equation}
P~ \subset~ (V~ \times~ \Sigma^*)~ \times C(\Sigma)~ \times~ (V~ \times~ E(\Sigma))^*
\end{equation}

\vspace{5mm} 

Where $(V~ \times~ \Sigma^*) $ is the predecessor module, $C(\Sigma) $ is the condition and $(V~ \times~ E(\Sigma))^* $ is the set of successor modules. For the sake of readability we can write out a production rule as \textit{predecessor} : \textit{condition} $\rightarrow$ \textit{successor}. I will be explaining the use of conditions in production rules in more detail in section \ref{Condition L-system Subsection}.\\
A module is said to match a production rule predecessor if they meet the three criteria below. In the case where the module does not match any of the production rule predecessors, the module is left unchanged, effectively rewriting itself. \\

\vspace{5mm}

$\bullet$ The name of the axiom module matches the name of the production predecessor. \\
$\bullet$ The number of parameters for the axiom module is the same as the number of parameters for the production predecessor. \\
$\bullet$ The condition of the production evaluates to true. If there is no condition, then the result is true by default.\\

\vspace{5mm}

\end{flushleft}

\subsection{Defining Constants and Objects}

There are some other features covered by Prusinkiewicz and Lindenmayer, that are not specific to the parametric L-systems definition itself but serve more as quality of life. In the literature, they refer to the \#define which is said "To assign values to numerical constants used in the L-system" as well as the \#include statement which specifies what type of shape to draw by refering to a library of predefined shapes \cite{prusinkiewicz2012algorithmic}. \\
For instance if we have an value for an angle that we would like to use within the production rules we can use the \#define statement as follows:

\vspace{5mm}

\begin{equation} \label{define statement example}
\begin{aligned}
	&n=4 \\
	&\textrm{\#define angle 90}\\
	&\omega~~ : F(5)\\
	&p_1~ :  F(x)~~~~~ :~ * \rightarrow~ F(w)+(angle)\\
\end{aligned}
\end{equation}

\vspace{5mm}

Here you can see that the \#define acts like a declaration, where we are going to be defining a variable which will be used later. Essentially we are replacing any occurences of the variable \textit{angle} with the value of 90 degrees. The define statement is written as  \#define \textit{variable\_name} \textit{value}. \\

\vspace{5mm}

With regards to the \#include statement, In the literature the \#include may be used by stating '\#include H'. This would tell the turtle interpreter that the symbol 'H' is a shape in a library of predefined shapes which should be rendered instead of the default shape. We have decided modify this functionality, instead of the \#include statement, we have provided the \#object statement. The \#object statement serves a similar purpose however instead of import the symbol 'H' we can specify, \#object H HETEROCYST, which specifies that we are associating the symbol or module 'H' with the object HETEROCYST. The HETEROCYST object will be stored in a predefined library. This way we can associate the same object to multiple symbols. It also does not limit us to a predefined name for an object. Below is an example using the \#object statement: \\

\vspace{5mm}

\begin{equation} \label{object statement example}
\begin{aligned}
	&n=1 \\
	&\textrm{\#object F BRANCH}\\
	&\textrm{\#object S SPHERE}\\
	&\omega~~ : F(1)\\
	&p_1~ :  F(x)~~~~~ :~ * \rightarrow~ F(w)F(w)F(w)F(w)S(w)\\
\end{aligned}
\end{equation}

\begin{figure}[htbp]
	{\centering
		\vspace{7px}
		\setlength{\fboxrule}{1pt}
		\fbox{
			\includegraphics[scale=0.2]{Diagrams/object_example.png}
		}
		\caption{Diagram of an L-system Using Multiple Objects.}
	}
\end{figure}
\FloatBarrier

\vspace{5mm}

I will be going into more detail about this and other features in section \ref{l-system generator section}.

\subsection{Modules With Special Meanings}

\begin{flushleft}


In the above section I defined the details of a parametric 0L-system, in the paper by Prusinkiewicz and Lindenmayer, there are two operators which I have not discussed yet, those are the ! and the ‘. Prusinkiewicz and Lindenmayer state that “The symbols ! and ‘ are used to decrement the diameter of segments and increment the current index to the color table respectively” \cite{prusinkiewicz2012algorithmic}. We have decided to modify this to work slightly differently, the ! and ‘ will still perform the same operation, however the ! and ‘ symbols are actually treated as a module that holds a particular meaning to the interpreter, rather than a single operator, furthermore, they share the same properties with modules, they can contain multiple parameters, and depending on the number of parameters they can be treated differently. The module ! with no parameters could mean decrement the diameter of the segment by a default amount, whereas !(10) means set the diameter of the segment to 10. The length can also be manipulated in a similar manner. The module with the name F has a default meaning to create a segment in the current direction by a default amount. If we provide the module F(10) we are specifying to create a segment of length 10.\\

\vspace{5mm}

Using the L-system below we can create figure \ref{parametric l-system practical}, the concepts discussed above have been used by decrementing the segment diameter during the rewriting process as well as by incrementing the branch length.

\vspace{5mm}

\begin{equation} \label{parametric l-system practical}
\begin{aligned}
	&n=8 \\
	&\omega~~ : A(5)\\
	&p_1~ :  A(w)~~~~~ :~ * \rightarrow~ F(1)!(w)[+A(w~*~0.707)][-A(w~*~0.707)]\\
	&p_2~ :  F(s)~~~~~ :~ * \rightarrow~ F(s~*~1.456)\\
\end{aligned}
\end{equation}

\vspace{5mm}

The above l-system gives the resulting representation shown below in figure 3.8. 

\begin{figure}[htbp]
	{\centering
		\vspace{7px}
		\setlength{\fboxrule}{1pt}
		\fbox{
			\includegraphics[scale=0.20]{ParametricLsystem/branchingPattern.png}
		}
		\caption{3D Parametric L-system.}
	}
\end{figure}

\FloatBarrier

This gives a much more realistic looking tree structure as the branch segments become shorter but also become thinner in diameter as they get closer to the end of the branch as a whole. 

\end{flushleft}



\subsection{Representing L-system Conditions} \label{Condition L-system Subsection}

\begin{flushleft}

A condition allows us to have multiple production rules that are the same in terms of the module name and the number of parameters that they have, furthermore, they require a particalar condition to be met in order for the module to match that rule. \\
In this section I will be detailing the use of the condition statement, which lies between the predecessor and the successor in a production rule, and can be seen as an a mathematical expression on either side of a relational operator. During the rule selection process the expressions are evaluated and the results are compared using the condition operator. If the result of the condition is true then that rule is selected for rewriting, if the result is false then it moves on to check the next rule. \\

\vspace{5mm}

Below is an example of a parametric 0L-system using condition statements:\\

\begin{equation} \label{parametric l-system example}
\begin{aligned}
	&n=5 \\
	&\omega~~ : A(0)B(0,4)\\
	&p_1~ :  A(x)~~~~~ :~ x~ >~ 2~ \rightarrow~ C\\
	&p_2~ :  A(x)~~~~~ :~ x~ <~ 2~ \rightarrow~ A(x~ +~ 1)\\
	&p_3~ :  B(x,~ y)~ :~ x~ >~ y~ \rightarrow~ D\\
	&p_4~ :  B(x,~ y)~ :~ x~ <~ y~ \rightarrow~ B(x~ +~ 1,~ y)\\
\end{aligned}
\end{equation}

\vspace{5mm}

The L-system above in \ref{parametric l-system example} is rewritten five times using the axiom specified by the symbol $\omega$, as well as the four production rules $p_1, p_2, p_3, p_4$. Each generation of the rewritting process can be seen below in \ref{parametric l-system example result}.

\vspace{5mm}

\begin{equation} \label{parametric l-system example result}
\begin{aligned}
	&g_0 :~ A(0)B(0,~4)\\
	&g_1 :~ A(1)B(1,~4)\\
	&g_2 :~ A(2)B(2,~4)\\
	&g_3 :~ C~B(3,~4)\\
	&g_4 :~ C~B(4,~4)\\
	&g_5 :~ C~D\\
\end{aligned}
\end{equation}

\vspace{5mm}

A practical use of the condition statement might be to simulate different stages of growth. This is best illustrated using the L-system below: \\

\vspace{5mm}

\begin{figure}[htbp]

\begin{equation} \label{conditional l-system example}
\begin{aligned}
	&n=2,~4,~6 \\
	&\#object~ F~ BRANCH \\
 	&\#object~ L~ LEAF \\
	&\#object~ S~ SPHERE \\
	&\#define~ r~ 45 \\
	&\#define~ len~ 0.5 \\
	&\#define~ lean~ 5.0 \\
	&\#define~ flowerW~ 1.0 \\
	&\omega~~ : !(0.1)I(5)\\
	&p_1~ :  I(x)~ :~ x~ >~ 0~~ \rightarrow~ F(len)-(lean)[R({0, 100})]F(len)[R({0, 100})]I(x-1)\\
	&p_2~ :  R(x)~ :~ x~ >~ 50~ \rightarrow~ -(r)/(20)!(2.0)L(2)!(0.1)\\
	&p_3~ :  R(x)~ :~ x~ <~ 50~ \rightarrow~ -(r)\backslash(170)!(2.0)L(2)!(0.1)\\
	&p_4~ :  I(x)~ :~ x~ <=~ 0~ \rightarrow~ F(len)!(flowerW)S(0.3)\\
\end{aligned}
\end{equation}

	{\centering
		\vspace{7px}
		\setlength{\fboxrule}{1pt}
		\fbox{
			\includegraphics[scale=0.13]{Diagrams/conditionalLsystem.png}
		}
		\caption{Condition Statements Used to Simulate the Growth of a Flower. 2nd Generation on the Left, 4th Generation in the Center and 6th Generation on the Right}
	}
\end{figure}

\FloatBarrier

\end{flushleft}


\subsection{Representing Randomness} \label{Randomness L-system Subsection}

\begin{flushleft}

Randomness is an essential part of nature. If there was no randomness in plant life, we would end up with very symetric and unrealistic plants. Randomness is also responsible for creating variation in the same L-system. A L-system essentially describes the structure and species of a plant. It describes everything from how large the trunk of the tree is, to how many leaves there are on the end of branch, or even if it has flowers or not. However if there is no capability to have randomness in the generation of the L-system then we will always end up with the exact same structure. 
\vspace{5mm}
Below is a simple example of how randomness can be used to create variation.

\end{flushleft}   

\begin{figure}[htbp]
	\raggedright
	\textbf{\underline{Random Fractal:}} \\
	\#n = 2; \\
	\#w : !(0.2)F(1.0); \\
	\#p1 : F(x) : * : F(x)[+(25)F(x)][-(25)F(x)]+(\{-20.0, 20.0\})F(x)-(\{-20.0, 20.0\})F(x);\\
	\vspace{10mm}
	{\centering
		\vspace{7px}
		\setlength{\fboxrule}{1pt}
		\fbox{
			\includegraphics[scale=0.20]{Diagrams/RandomTrees.png}
		}
		\caption{Different Variations of the Same L-system with Randomness Introduced in The Angles. \label{figRandomness}}
	}
\end{figure}
\FloatBarrier

\begin{flushleft}

In figure \ref{figRandomness} there are four variations of the same L-system using randomness, We can specify that we would like to create a random number by using the expression \{-20.0, 20.0\}. The curly braces signify that what is contained is a random number range, ranging from the minimum value as the first floating point value and the maximum value as the second floating point value separated by a comma. If both values are the same for instance +(\{10.0, 10.0\}) this is equivilant to +(10.0).

\end{flushleft}

\subsection{Stochastic Rules in the L-system} \label{Stochastic L-system Subsection}

\begin{flushleft}

\begin{figure}[htbp]
	{\centering
		\vspace{7px}
		\setlength{\fboxrule}{1pt}
		\fbox{
			\includegraphics[scale=0.35]{Diagrams/stochastics.png}
		}
		\caption{Representation of an L-system with a probability stochastic with a 33.33\% chance for each rule.}
	}
\end{figure}

\FloatBarrier


\end{flushleft}






