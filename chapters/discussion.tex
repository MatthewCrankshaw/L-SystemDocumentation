
The relationship between the string rewriting system and the the string interpreter system cannot be independant of one another. As complexity is added one system the other need not be as complex. This can be described with the simple example of determining the branch width of each branch segment. In one case the branch width could be determined within the L-system rewriter by decrementing the branch width within the production rules. On the other hand you could leave the process of determining the brach width to the interpreter. This may require the interpreter understanding where in the tree the branch lies as well as information as to the base width as well as the rate at which the branches decrease in size. There are arguements that can be made for both sides of this discussion, and it can be difficult to determine where the line of complexity should lie between the rewriter and the interpreter. Depending on what the L-system is representing, there may be a need for emphasis on one or the other side. This implementation contained within in this thesis puts emphasis on providing a large amount of information in an interpreter independant way. this means that the information is provided through parameters of modules. If more specific instructions are required they can be provided through the use of \#object defined modules which will point to a specific meaning defined within the interpreter. This allows an L-system to provide specific information to the interpreter without having the grammar of the L-system dictate how it should be interpreted.