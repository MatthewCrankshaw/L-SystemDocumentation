The relationship between the string rewriter and the string interpreter systems cannot be independent of one another. The rewriter is responsible for creating the overall structure and information for the interpreter. The interpreter takes this information and models it to the best of its ability. They are systems that share complexity. If one system is highly complex, the other one need not be as complex. The goal is to create a procedural generation system, where a user must specify a plants' structure using the L-system language. The L-system, and therefore the rewriter, cannot become so complex that it becomes unrealistically challenging to write an L-system for a plant. It is also not reasonable to create an L-system that is overly simplistic, like a DOL-system, because then the interpreter will need to assume certain features of the plant.  It can be challenging to determine where the line of complexity should lie between the rewriter and the interpreter. Depending on the L-system being represented, there may be a need for emphasis on either side. The implementation contained within this thesis puts emphasis on providing a large amount of information in a way that is interpreter independent. This means that the information is provided through the parameters of modules and through declarations like the \#object declaration. If more specific instructions are required, they can be provided through the use of the \#object defined module, which will point to a meaning defined within the interpreter. This allows an L-system to provide specific information to the interpreter without the L-system dictating how it should be interpreted.

The advantage of this approach is the L-system grammar does not need to change, regardless of how it is interpreted. Additionally, the rewriting process is also kept independent of the interpretation. The interpretation of a particular module is defined by the interpreter, but can also be modified by the L-system using specific modifiers. For instance, the module name \say{F} can be interpreted as a turtle graphics instruction to move forward. However, the statement \say{\#object F BRANCH} can be used to modify the meaning of this within the interpreter. Such that the meaning now suggests a move forward while also indicating that a BRANCH object should be rendered at that position. This makes it clear to the interpreter and the person writing the L-system, what is going to be rendered.