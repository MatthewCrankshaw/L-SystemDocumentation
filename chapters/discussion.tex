\lettrine[lines=3]{T}{}his thesis investigated whether a procedural generation system can produce both the model and physical properties necessary to render and simulate realistic looking plant-life. The parametric L-system provides a way of defining the structure of a plant as well as parameters that can represent additional information such as the physical properties of the plant. The parameters can be manipulated during the rewriting process to provide multiple different effects, discussed in chapter \ref{results chapter}, such as the width of a tree's branch increasing exponentially with the number of generations. The parametric L-system is a compelling tool for the procedural generation of plant-life as it can produce realistic looking plant structures very quickly and efficiently. In modern 3D applications, a large part of what users perceive as realistic has to do with the movement and motion of objects within a scene. For instance, detailed tree model will appear unrealistic if it is completely motionless in the middle of a storm. Having a single procedural generation system that not only creates the plant model but can provide the skeletal structure and physical attributes it is very convenient as it encapsulates the entire description in one place.

The implementation of the procedural generator for this thesis has two major systems; the L-system rewriter and the interpreter. The rewriter and the interpreter operate independently of one another, however to create the desired result they must both cooperate. Each system has a level of complexity that relies on the other. For instance, a rewriter with many sophisticated features will provide more information to the interpreter; therefore, the interpreter can follow the instructions precisely to produce the desired result. Conversely, a simple rewriter with few features will provide less information to the interpreter, and the interpreter will then have to make more assumptions and do more work to achieve the same result. For the procedural generation system to be effective, the L-system grammar cannot become so complicated that it is unreasonabily difficult to write an L-system for a given plant. It is also limiting to create an L-system that is overly simplistic, such as a DOL-system, as the plant may become too dependant on the interpreter, and become inflexible. It can be challenging to determine where the line of complexity should lie between the rewriter and the interpreter. Depending on the L-systems' representation, there may be a need for emphasis on one side rather than the other. For the implementation in this thesis, the parametric L-system focuses more on the complexity within the rewriter. Having features like parameters, stochastic rules, and conditions mean that the L-system is very powerful, and can provide a large portion of information to the interpreter for both rendering and simulating plants. However, this does have a drawback that the L-system becomes challenging to write and understand.

Although writing the L-systems may be more challenging, changing the appearance of a plant actually becomes more straightforward. As shown in chapter \ref{results chapter}, parameters such as the angle, width, or spring constant of a branch can have a dramatic effect on the final visual result of the plant without changing the structure of the L-system. This is further improved with features like stochastic rules and random ranges, which makes it possible to produce models with variation in the structure. These features could be used by artists to create many different variations of the same family of plant-life from the same L-system description.

Prompts can also be provided to the interpreter to render particular objects or effects by using the \#object declaration. Examples of these features can be seen in chapter \ref{l-system chapter}. This allows an L-system to provide specific information to the interpreter without the L-system dictating how it should be interpreted. The advantage of this approach is the L-system rewriter does not need to change, regardless of the interpretation. Additionally, the rewriting process is kept independent of the interpretation. This allows features to be added to the interpreter and simulator without affecting the rewriting system.

There are several ways the creation of plants using an L-system could be made more accessible for those who are not familiar with writing L-systems. One option would be to create a real-time tool where a user can edit the L-system and have its representation reflected immediately as a generated and simulated plant. This would give a user immediate feedback about the effects of the parameter changes and whether they are producing the desired outcome. Including animation of the plant in real-time would allow the user to see the plants behaviour under different wind or gravitational conditions. A different option may be to have a large number of predefined L-systems and only give the user control over manipulating the parameters of the L-system. This may give the user less control over the `species' of the plant but will make it much easier to modify the plant and get a result quickly.

The parametric L-system and interpreter work together as a system, taking in the L-system file and some additional information about the physics simulation, and outputting the model data, such as plant skeleton and model vertices. This allows the L-system and interpreter system to be created in the form of a plug-in or sub-system. Many game engines like Unreal Engine and Unity as well as specialised 3D applications such as Autodesk Maya and Blender support plug-ins or separate sub-systems that provide more domain-specific functionality. The Unreal Engine supports many plug-ins that add specialised tools such as advanced audio functionality, realistic and interactive water, or even specialised tools for creating combat animations. The advantage of building this functionality as a plug-in or sub-system is that it can use existing functionality within that game engine or 3D application. For instance, an Unreal Engine plug-in may make use of the PhysX 3.3 physics engine or the newer Chaos Physics Engine to handle all of the plant's physics in an optimised way. 
\newpage
It has been shown that parametric L-systems can be used to provide the information necessary to simulate the effect of gravity and wind on a procedurally generated plant. Using the L-system to specify features such as the width, length, and bending coefficient of branches makes it relatively straightforward to implement a physical simulation within the interpreter. As these features are used for the simulation, the simulator is flexible enough to work on many different types of plant-life. There is also an understandable relationship between the changes to the parameters and the effects they have on the resulting plant model.
