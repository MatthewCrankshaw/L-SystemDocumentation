
The relationship between the string rewriting system and the the string interpreter system cannot be independant of one another. As complexity is added one system the other need not be as complex. This can be described with a simple example of determining the branch width of each segment. On one hand the branch width could be determined within the L-system rewriter by decrementing the branch width within the production rules. On the other hand you could leave the process of determining the branch width to the interpreter. This may require the interpreter to  understand where the branch lies within the tree, as well as information as to the base width and rate at which the branches decrease in size. There are arguements that can be made for both sides of this discussion. It can be difficult to determine where the line of complexity should lie between the rewriter and the interpreter. Depending on what the L-system is representing, there may be a need for emphasis on one or the other side. The implementation contained within this thesis puts emphasis on providing a large amount of information in an interpreter independant way. This means that the information is provided through the parameters of modules. If more specific instructions are required they can be provided through the use of a \#object defined module, which will point to a meaning defined within the interpreter. This allows an L-system to provide specific information to the interpreter without the grammar of the L-system dictating how it should be interpreted.

The advantage to this approach is that the L-system grammar does not need to change regardless of interpretation, nor does the rewriting takes process. The interpretation of a particular module is definined by the interpreter itself, but can also be modified by the L-system. For instance, the module name \say{F} can be interpreted as a turtle graphics instruction to move forward. However, the statement \say{\#object F BRANCH} can be used to modify the meaning of this within the interpreter. Such that the meaning now suggests a move forward whilst also indicating that a BRANCH object should be rendered at that position. This makes it clear not only to the interpreter but also to the person writing the L-system.