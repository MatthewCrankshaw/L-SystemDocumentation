This thesis set out to explore the relationship between the L-system sting rewriter and the interpreting system, which work together to produce rendered models of plant-life. Furthermore, this thesis sought to create a parametric L-system that is not only capable of producing a plant's structure but also procedurally generates the parameters necessary to physically simulate any plant generated.  

The string rewriter and the string interpreter cannot operate entirely independently of one another. The rewriter is responsible for creating the overall structure and information of a plant. The output string of the rewriter is used as the input of the interpreter; each module within the string contains information that is used to generate and simulate the plant. These two systems each have a level of complexity that relies on the other. A rewriter with many sophisticated features will provide more information to the interpreter, meaning that the interpreter can follow the instructions to produce the result. Conversely, a simple rewriter with few features will provide less information to the interpreter, and therefore the interpreter will have to make more assumptions and do more work to achieve the same result. The goal is to create a procedural generation system, where a user must specify a plants' structure using the L-system language. The L-system, and therefore the rewriter, cannot become so complex that it becomes unrealistically challenging to write for a given plant. It is also limiting to create an L-system that is overly simplistic, like a DOL-system, as the plant may become too dependant on the interpreter, and become inflexible. It can be challenging to determine where the line of complexity should lie between the rewriter and the interpreter. Depending on the L-system being represented, there may be a need for emphasis on one side rather than the other.

The implementation contained within this thesis focused on providing parameters for each instruction within the L-system; those parameters contain information that is useful for the interpreter. The interpreter may choose not to use this information. For instance, if parameters for physics are provided, but the result is not going to be physically simulated. The parametric L-system is further improved by the use of rule conditions, stochastic rules, and random ranges, making it possible to create growth stages or variations in a plants' structure. Prompts can also be provided to the interpreter to render particular objects or effects by using the \#object declaration. This allows an L-system to provide specific information to the interpreter without the L-system dictating how it should be interpreted. The advantage of this approach is the L-system rewriter does not need to change, regardless of the interpretation. Additionally, the rewriting process is kept independent of the interpretation.

The L-system grammar is a compelling tool for the procedural generation of plant-life, it is possible to make very realistic looking plant structures very quickly and efficiently. In modern 3D applications, particularly video games, a large part of what users perceive as realistic has to do with the movement and motion of objects within a scene. For instance, a compelling tree model can be made to look unrealistic if it is completely motionless in the middle of a storm. Having the L-system provide the parameters for physical simulation is very powerful; it allows a single procedural generation system to not only create the plant model but simulate in a short L-system. The main disadvantage of L-systems is how difficult it is to write the L-system language to produce the desired result. The L-system is essentially a grammar similar to that of a programming language. It becomes very challenging to define a particular plant as an L-system. The average artist or even programmer would take a while understanding how to get the results they desire through the L-system grammar. 

There are several ways to improve the creation of plants using an L-system. One improvement might be to create a tool where a user can edit the L-system and have its representation reflected immediately as a generated plant. This will give immediate feedback about whether they are getting the desired outcome. This can be built to animate the plant in real-time to see how it would react to different forces like gravity or wind. A different option may be to have a large number of predefined L-systems and only give the user control over manipulating the parameters of the L-system. This may give the user less control over the look of the plant itself but will make it much easier to modify the plant and get a result quickly.

This thesis found that it is possible to use and L-system to provide the information necessary to simulate the effect of gravity and wind on a procedurally generated plant. Using the L-system to specify features such as the width, length, and bending coefficient of branches makes it relatively straightforward to implement a physical simulation within the interpreter. In the same way that there is a trade-off between the rewriter and interpreter for procedural generation, the features provided for simulation makes the L-system more cumbersome and difficult to read. This begs the question as to what features should be provided by the L-system and which should be left up to the interpreter to decide? For instance, should the density of the plant material be defined by each branch segment within the L-system, or should it be chosen by the interpreter? 
