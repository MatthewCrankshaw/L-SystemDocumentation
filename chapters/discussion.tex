The string rewriter and the string interpreter cannot operate entirely independently of one another. The rewriter is responsible for creating the overall structure and information for the interpreter. The interpreter takes this information and models it to the best of its ability. They are systems that share complexity. If one system is highly complex, the other one need not be as complex. The goal is to create a procedural generation system, where a user must specify a plants' structure using the L-system language. The L-system, and therefore the rewriter, cannot become so complex that it becomes unrealistically challenging to write an L-system for a given plant. It is also not reasonable to create an L-system that is overly simplistic, like a DOL-system, as then the interpreter will need to assume certain features of the plant.  It can be challenging to determine where the line of complexity should lie between the rewriter and the interpreter. Depending on the L-system being represented, there may be a need for emphasis on one side rather than the other.

The implementation contained within this thesis emphasizes providing a large amount of information in a way that is interpreter independent. This means that the information is provided through the parameters of modules and declarations like the \#object declaration. If more specific instructions are required, they can be provided through the use of the \#object defined module, which will point to a meaning defined within the interpreter. This allows an L-system to provide specific information to the interpreter without the L-system dictating how it should be interpreted. The advantage of this approach is the L-system rewriter does not need to change, regardless of the interpretation. Additionally, the rewriting process is also kept independent of the interpretation. The meaning of a particular module is defined by the interpreter, but the meaning can be provided by modifying the L-system using the \#object declaration. 

The L-system grammar is a compelling tool for the procedural generation of plant-life, it is possible to make very realistic looking plant structures very quickly and efficiently. The main disadvantage of the L-system has to do with writing the L-systems. The L-system is essentially a grammar similar to that of a programming language. It becomes challenging to define a particular plant as an L-system. The average artist or even programmer would take a while understanding how to get the results they desire through the L-system grammar. There are several ways to improve the creation of plants using an L-system. One improvement might be to create a tool where a user can edit the L-system and have its representation reflected immediately as a generated plant. This will give immediate feedback about whether they are getting the desired outcome. This can be built to animate the plant in real-time to see how it would react to different forces like gravity or wind. A different option may be to have a large number of predefined L-systems and only give the user control over manipulating the parameters of the L-system. This may give the user less control over the look of the plant itself but will make it much easier to modify the plant and get a result quickly.

This thesis found that it is possible to use and L-system to provide the information necessary to simulate the effect of gravity and wind on a procedurally generated plant. Using the L-system to specify features such as the width, length, and bending coefficient of branches makes it relatively straightforward to implement a physical simulation within the interpreter. In the same way that there is a trade-off between the rewriter and interpreter for procedural generation, the features provided for simulation makes the L-system more cumbersome and difficult to read. This begs the question as to what features should be provided by the L-system and which should be left up to the interpreter to decide? For instance, should the density of the plant material be defined by each branch segment within the L-system, or should it be chosen by the interpreter? 
