This thesis investigated whether a procedural generation system can produce both the model and physical properties necessary to render and simulate realistic looking plant-life. The parametric L-system provides a way of defining the structure of a plant as well as parameters that can be used to hold additional information such as the physical properties of the plant. The parameters can be manipulated during the rewriting process to provide different effects, as discussed in chapter \ref{results chapter}, where the width of a tree's branch increases exponentially with the number of generations. The parametric L-system is a compelling tool for the procedural generation of plant-life as it can produce realistic looking plant structures very quickly and efficiently. In modern 3D applications, a large part of what users perceive as realistic has to do with the movement and motion of objects within a scene. For instance, a compelling tree model can be made to look unrealistic if it is completely motionless in the middle of a storm. Having a single procedural generation system that not only creates the plant model but can provide the skeletal structure and information to physically simulate it is very convenient.

The implementation of the procedural generator for this thesis has two major systems. These are the L-system rewriter and the interpreter. The rewriter and the interpreter cannot operate entirely independently of one another. Each system has a level of complexity that relies on the other. For instance, a rewriter with many sophisticated features will provide more information to the interpreter; therefore, the interpreter can follow the instructions precisely to produce the desired result. Conversely, a simple rewriter with few features will provide less information to the interpreter, and the interpreter will then have to make more assumptions and do more work to achieve the same result. For the procedural generation system to be effective, the L-system grammar cannot become so complicated that it is difficult to write an L-system for a given plant. It is also limiting to create an L-system that is overly simplistic, like a DOL-system, as the plant may become too dependant on the interpreter, and become inflexible. It can be challenging to determine where the line of complexity should lie between the rewriter and the interpreter. Depending on the L-systems' representation, there may be a need for emphasis on one side rather than the other. For the implementation in this thesis, the parametric L-system focuses more on the complexity within the rewriter. Having features like parameters, stochastic rules, and conditions mean that the L-system is very powerful, and can provide a large portion of information to the interpreter for both rendering and simulating plants. However, this does have a drawback that the L-system becomes challenging to write and understand.

Although writing the L-systems may be more challenging, changing the look of a plant actually becomes more straightforward. As shown in chapter \ref{results chapter}, changing parameters such as the angle, width, or spring constant of a branch can have a dramatic effect on the overall look of the plant without having to change the structure of the L-system at all. This is further improved with features like stochastic rules and random ranges, which makes it possible to produce models with variation in the structure. These features could be used by artists to create many different variations of the same family of plant-life in a matter of seconds.

Prompts can also be provided to the interpreter to render particular objects or effects by using the \#object declaration. Examples of these features can be seen in chapter \ref{l-system chapter}. This allows an L-system to provide specific information to the interpreter without the L-system dictating how it should be interpreted. The advantage of this approach is the L-system rewriter does not need to change, regardless of the interpretation. Additionally, the rewriting process is kept independent of the interpretation.

There are several ways the creation of plants using an L-system can be made more accessible for those who are not comfortable with writing L-systems. One improvement might be to create a tool where a user can edit the L-system and have its representation reflected immediately as a generated and simulated plant. This will give a user immediate feedback about whether they are getting the desired outcome. This can be built to animate the plant in real-time to see how it would react to different forces like gravity or wind. A different option may be to have a large number of predefined L-systems and only give the user control over manipulating the parameters of the L-system. This may give the user less control over the `species' of the plant but will make it much easier to modify the plant and get a result quickly.

This thesis found that it is possible to use and L-system to provide the information necessary to simulate the effect of gravity and wind on a procedurally generated plant. Using the L-system to specify features such as the width, length, and bending coefficient of branches makes it relatively straightforward to implement a physical simulation within the interpreter. In the same way that there is a trade-off between the rewriter and interpreter for procedural generation, the features provided for simulation makes the L-system more cumbersome and difficult to read. This begs the question as to what features should be provided by the L-system and which should be left up to the interpreter to decide? For instance, should the density of the plant material be defined by each branch segment within the L-system, or should it be chosen by the interpreter? 
