\section{Language and environment}

To understand what programming language and environment will be best suited for this project, I first  provide the technical requirements that will need to be met. The programming language will need to be  relatively fast when processing the rules of the L-systems and when rewriting these strings according to those rules. \\
\\
The program will also need to interpret the strings generated by the L-system rules and be able to generate a 3 dimensional representation of that L-system. This representation will need to be intuative and will have to allow me to examine it from different perspectives. In order to make the representation intuative, the 3D representation should be rendered in real time at multiple frames per second. And the user should be able to use a computer mouse and keyboard to move around the 3D world. \\ 
\\
Due to these demands have decided to use the C and C++ programming language. Due to its and thoroughly tested in built standard template library, I can count on it to be reliable and fast enough for the purpose of this project. It will also allow me to use other very useful libraries for 3D graphics such as OpenGL which is also written in C/C++. I will be speaking in more detail about these details of these libraries in later sections. \\
\\
In order to create a window and provide the environment for writing pixels to the screen I will make use of the Graphics Library Framework (GLFW). Some useful mathematics functions and facilities can be found in the OpenGL Matematics Library (GLM) and possibly the most important for rendering in 3D is the Open Graphics Library or OpenGL for short. All of these libraries together will provide me with a strong foundation of tools that I can use to approach the practical aspect of this project. \\
\\
 

\subsection{C/C++ Programming Language}

The C programming language developed by Dennis Richie and Bell Labs in 1972 has been one of the most popular programming languages for a number of decades now \cite{ritchie1975c}. The C language was then extended upon by Bjarne Stroustrup in 1985 to create the C++ programming language \cite{stroustrup2000c++}. I have included both C and C++ as the programming languages that I will be using, as they are very closely related and C code can be compiled using the C++ compiler. For the most part I will be writing C++ code and making use of its object oriented features. However, their are instances when it will be more convenient to write C code or make use of a C library. 

\subsection{Standard Template Library (STL)}

The STL in C/C++ provides a number of useful functions, data structures and algorithms that have been extensively tested for both reliability and efficiency. The most common features I will be using are strings, vectors, stacks and input and output. It is possible that in some cases it may be more efficient to use custom data structures for the most part the STL functions will be more than good enough. \cite{horton2015stl} 

\subsection{Open Graphics Library (OpenGL)}

OpenGl is a 2D and 3D graphics API
\cite{movania2017opengl}

\subsection{OpenGL Mathematics Library (GLM)}

\subsection{Graphics Library Framework(GLFW)}

\subsection{Git Version Control}