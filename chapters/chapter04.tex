
\begin{flushleft}

In order to generate realistic models of trees, there are a number of characteristics which are needed in order to provide a realistic looking tree. The motion of trees is not something really think too much about on an everyday basis. A plants movement is very subtle, however, when completely devoid of movement, the plant starts looking very unnatural. We will now be looking into simulating the animation of plants, and building this functionality into the parametric L-system which we covered in previous sections. Not all 3D applications may need to make use of simulating the plants, so the solution will allow us to declare the parameters of a 3D animation, or if not neccessary, leave them out entirely.\\

\vspace{5mm}

\cite{barron2001real}

\end{flushleft}

\section{Motion Equations}

Torque\\
$ \tau = I\alpha $ \\
Where $\tau$ is the torque, $I$ is the moment of inertia and $\alpha$ is the angular acceleration.\\
\\
$ \tau = f \otimes R $ \\
Where $\tau$ is the torque, $f$ is the force acting on the end of the branch and $R$ is the vector representing the length and orientation of the branch. \\
\\
Mass \\
$ M = \Pi~ r ^ 2 h $  \\
Where $M$ is the mass represented in kg, $r$ is the radius of the branch in meters and $h$ is the height of the branch in meters. \\
\\
Inertia\\
$ I = \frac{1}{3} M L ^ 2 $ \\ 
Where $I$ is the inertia of the branch, $M$ is the Mass of the branch and $h$ is the height of the branch. \\
\\
Angular Acceleration\\
$ \omega = \omega _0 \alpha t $ \\
Where $\omega$ is the angular velocity, $\omega _0 $ is the previous angular velocity, $\alpha$ is the angular acceleration and $t$ is the change in time. \\
\\
Next angle equation\\
$ \theta = \theta _0 + \omega _0 t + \frac{1}{2} \alpha t ^2 $ \\
Where $\theta$ is the angle, $\theta _0$ is the previous angle, $w _0$ is the previous angular velocity, t is the change in time and $\alpha$ is the angular acceleration. \\
\\

\section{Hook's Law}

$ f = -k _s x + k _d v$\\
Where $f$ is the force exerted by the spring, $k _s$ is the spring constant and $x$ is the total displacement of the spring. We then would also like to add a dampening force which is the $k _d v$ part where $k _d$ is the dampening constant and $v$ is the velocity at the end of the spring.\\
