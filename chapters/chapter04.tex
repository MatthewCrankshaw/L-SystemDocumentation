

\lettrine[lines=3]{T}{}he motion of plants is an important factor when looking to create realistic looking plant-life. It has been a topic of discussion and research for many years now, particularly with regards to grass, bushes and trees within video games. The movement is usually very subtle, but if it is completely missing a scene can start looking very unnatural, making the user feel uncomfortable. This chapter will discuss a method of simulating the physical motion of plant-life layed out by Barron et al \cite{barron2001real}. This method will be built into the parametric L-system itself in such a way that the L-system can provide the physical parameters for the simulation. This will allow a physics simulation to be run on any plant generated by the L-system.

The main technique discussed by Barron et al for simulating the motion of a system like a tree or plant, is taken from that of a particle system first described by Reeves \cite{reeves1983particle}. Particle systems can be applied to simulate phenomena like clouds, smoke, water and fire. The main advantage of particle systems is that their motion can be updated simaltaniously. This technique can be applied to the L-system representation of plant-life, where branches are split into segments making up a skeleton of joints. Each joint can represent a \say{particle} within the system, which has a dependency on all parent branches.

\section{Motion Equations}

Torque\\
$ \tau = I\alpha $ \\
Where $\tau$ is the torque, $I$ is the moment of inertia and $\alpha$ is the angular acceleration.\\
\\
$ \tau = f \otimes R $ \\
Where $\tau$ is the torque, $f$ is the force acting on the end of the branch and $R$ is the vector representing the length and orientation of the branch. \\
\\
Mass \\
$ M = \Pi~ r ^ 2 h $  \\
Where $M$ is the mass represented in kg, $r$ is the radius of the branch in meters and $h$ is the height of the branch in meters. \\
\\
Inertia\\
$ I = \frac{1}{3} M L ^ 2 $ \\ 
Where $I$ is the inertia of the branch, $M$ is the Mass of the branch and $h$ is the height of the branch. \\
\\
Angular Acceleration\\
$ \omega = \omega _0 \alpha t $ \\
Where $\omega$ is the angular velocity, $\omega _0 $ is the previous angular velocity, $\alpha$ is the angular acceleration and $t$ is the change in time. \\
\\
Next angle equation\\
$ \theta = \theta _0 + \omega _0 t + \frac{1}{2} \alpha t ^2 $ \\
Where $\theta$ is the angle, $\theta _0$ is the previous angle, $w _0$ is the previous angular velocity, t is the change in time and $\alpha$ is the angular acceleration. \\
\\

\section{Hook's Law}

$ f = -k _s x + k _d v$\\
Where $f$ is the force exerted by the spring, $k _s$ is the spring constant and $x$ is the total displacement of the spring. We then would also like to add a dampening force which is the $k _d v$ part where $k _d$ is the dampening constant and $v$ is the velocity at the end of the spring.\\
