

\lettrine[lines=3]{T}{}he motion of plants is an important factor when looking to create realistic looking plant-life. It has been a topic of discussion and research for many years now, particularly with regards to grass, bushes and trees within video games. The movement is usually very subtle, but if it is completely missing a scene can start looking very unnatural, making the user feel uncomfortable. This chapter will discuss a method of simulating the physical motion of plant-life layed out by Barron et al \cite{barron2001real}. This method will be built into the parametric L-system itself in such a way that the L-system can provide the physical parameters for the simulation. This will allow a physics simulation to be run on any plant generated by the L-system.

The main technique discussed by Barron et al for simulating the motion of a system like a tree or plant, is taken from that of a particle system first described by Reeves \cite{reeves1983particle}. Particle systems can be applied to simulate phenomena like clouds, smoke, water and fire. The main advantage of particle systems is that their motion can be updated simaltaniously. This technique can be applied to the L-system representation of plant-life, where branches are split into segments making up a skeleton of joints. Each joint can represent a \say{particle} within the system, which has a dependency on all parent branches.

Using the concept of a particle system, the motion of the plant can simulated by having each joint within the plant skeleton to be seen as a spring consistant with Hook's Law. The spring force of the branch tries to prevent it from bending. Whereas gravity, wind and other forces cause torque, which generally acts against this spring force, causing the branch to bend.

The mass of each branch segment can be simply calculated by taking its radius $r$ and the length $l$ and applying the following equation.

\begin{equation}
m  = \pi r^2 l
\end{equation}

\noindent
Where $m$ is the mass represented in kg, $r$ is the radius of the branch in meters and $l$ is the length of the branch in meters. Using this we can calculate the segments moment of inertia, the branch can be simply as a long thin cylinder.

\begin{equation}
I = \frac{1}{3} m l ^ 2
\end{equation}

\noindent
Where $I$ is the inertia of the branch, $m$ is the Mass of the branch and $l$ is the lenght of the branch. Similarly an inertia tensor can be used for the sake of convenience and to better describe the objects rotational inertia. 

\begin{equation}
\begin{aligned}
I = \begin{bmatrix}
\frac{1}{12}m(3r^2 + h^2) 	& 0 							& 0 \\
0 							& \frac{1}{12}m(3r^2 + h^2)		& 0 \\
0 							& 0 							& \frac{1}{2}mr^2 
\end{bmatrix}
\end{aligned}
\end{equation}

\noindent
The forward vector, this being the vector in the direction that the branch is pointing towards, can be used to calcuate the force vector acting on the branch $V$, by taking the cross product of the forward  vector $v$ and the force vector $w$.

\begin{equation}
V = v \otimes w
\end{equation}

\noindent
The dispacement can be calculated by keeping track of the starting local rotation of the branch $p$ as well as the current rotation of the branch $q$ in the form of two quaternions. We can then calculate a quaternion $d$ for the differece of the two quaternions, by taking $p$ and multiplying it by the inverse of $q$. 

\begin{equation}
d = p \times q^{-1}
\end{equation}

\noindent
Using the displacement vector of the branch we can apply it to Hook's Law, to get the force of the spring. 

\begin{equation}
f = -k _s d + k _d v
\end{equation}

\noindent
Where $f$ is the force exerted by the spring, $k _s$ is the spring constant and $x$ is the total displacement of the spring. The dampening force can be calculated as $k _d v$ part where $k _d$ is the dampening constant and $v$ is the velocity at the end of the spring. The forces can then be multiplied together to get the net force $f_net$ acting on the spring, this can be used to calculate the momentum and furthermore the velocity of the the branch. $T_{delta}$ is that change in time between physics calculations.

\begin{equation}
M = M_0 + f_{net} * T_{delta}
\end{equation}

\noindent
The velocity $v$ can be calculated by taking the inverse of the inertia tensor $I$ and multiplying that by the momentum vector $M$.

\begin{equation}
v = I^{-1} * M
Q_v = [0, v]
\end{equation}

\noindent
The velocity vector can be converted to its quaternion form $Q_v$ in order to make the last step simpler. The scalar part of quaternion can be set to 0 and the vector part can be set to $v$. This allows the next rotation quaternion $R$ to be calculated. 

\begin{equation}
R = R_0 + (\frac{1}{2} * Q_v * R_0 * T_{delta})
\end{equation}

\noindent
Where $R$ is the next local rotation quaternion, $R_0$ is the previous local rotation quaternion, $Q_v$ is the velocity quaternion and finally $T_{delta}$ is the change in time since the previous physics update.

