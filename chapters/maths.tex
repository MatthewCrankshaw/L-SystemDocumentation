
In any 3D application, mathematical models are used to represent the positions, rotations, and scale of objects within a given scene. It is crucial for this thesis to briefly touch on some of the core concepts, particularly for representing and manipulating 3D objects. 

All objects within a 3D application are made up of a set of vertices or points, which are represented with X, Y, and Z coordinates. Three vertices can make up one triangle, also called a face, multiple faces will then make up a whole 3D object. The use of mathematical methods in 3D graphics is to manipulate the vertices within an object consistently.  These methods include: rotating, translating, or scaling objects within a scene. 

This section will provide sufficient background on some of the essential concepts of 3D Mathematics, such as vectors, matrices, and quaternions, that are used widely in the turtle graphics interpreter as well as the model generator.

\section{Vectors}

Vectors have many meanings in different contexts, in \acrshort{3d} computer graphics, vectors often refer to the Euclidean vector. The Euclidean vector is a quantity in $n$-dimensional space that has both magnitude and direction. Vectors can be represented as a line segment pointing in a direction, with a certain length. A \acrshort{3d} vector can be written as a triple of scalar values eg: $(x, y, z)$.

The most common operations on vectors are multiplication by a scalar, addition, subtraction, normalisation and the dot and cross product. The multiplication by a scalar value can be seen as scaling the magnitude of the vector. This operation can be done uniformly or non-uniformly, as seen in the equation below:

\begin{equation}
a \otimes s = (a_x s_x, a_y s_y, a_z s_z)
\end{equation}

\noindent
Where $\otimes$ is the component-wise product of vector $a$, and the scaling vector $s$. Similar to the scalar product of a vector, the addition and subtraction of two vectors are the component-wise sum or difference. The equation for the sum and difference of a vector with a scalar value can be seen below. 

\begin{equation}
\begin{aligned}
a \oplus b = [(a_x + b_x), (a_y + b_y), (a_z + b_z)]\\
a \ominus b = [(a_x - b_x), (a_y - b_y), (a_z - b_z)]
\end{aligned}
\end{equation}

\begin{figure}[htbp]
	{\centering
		\setlength{\fboxrule}{1pt}
		\vspace{7px}
		\fbox{
			\includegraphics[scale=0.2]{Diagrams/vector_addition.png}
			\label{3DAxisFigure}
		}
		\caption{Table of common dot product tests between two vectors.}
	}
\end{figure}
\FloatBarrier

\noindent
A type of vector that is used very often in 3D graphics is known as a unit vector. This is a vector that has a magnitude of 1. Unit vectors are used extensively, particularly with \gls{Shader}s. Take the vector $v$ its magnitude $\alpha$ can be calculated by taking the square root of the product of its components squared, as seen below. 

\begin{equation}
	\alpha =~ \mid \textbf{v} \mid~ = \sqrt{\textbf{v}^2_x + \textbf{v}^2_y + \textbf{v}^2_z}
\end{equation}

The unit vector can then be calculated by taking the product of $v$ and the inverse of its magnitude shown in the following equation.

\begin{equation}
	\upsilon = \frac{\textbf{v}}{\alpha} = \frac{1}{\alpha} \textbf{v}
\end{equation}

There are many different ways to multiply vectors. The two main multiplications being the dot and cross product. The dot product yields a scalar value by adding the products of the vector product components. The cross product, on the other hand, is the product of two vectors, which gives a vector that is perpendicular. The dot product can be calculated using the formula below.

\begin{equation}
a \cdot b = a_x b_x + a_y b_y + a_z b_z = d
\end{equation}

\noindent
Some of the primary uses for dot products within 3D graphics is to find whether two vectors are collinear, perpendicular, or if they are in the same direction or opposite directions. One possible use for this is to find if two branches are growing in the same direction or in opposite directions. In the table \ref{dot product test} below, there are all of the dot product tests as well as its equation. Please note that $ab = \mid a \mid \mid b \mid = a \cdot b$.

\begin{table}[h!]
\centering
\begin{tabular}{ | c | c | c |}
\hline
	Test 	& Equation & Example\\  
\hline
\hline
	Collinear 							& $(a \cdot b) = ab$ & \includegraphics[scale=0.1]{Diagrams/vector1.png}\\
\hline
	Opposite Collinear 					& $(a \cdot b) = -ab$ &	\includegraphics[scale=0.1]{Diagrams/vector2.png}\\
\hline
	Perpendicular 						& $(a \cdot b) = 0$	&\includegraphics[scale=0.1]{Diagrams/vector3.png}\\
\hline
	Same Direction 						& $(a \cdot b) > 0$ &\includegraphics[scale=0.1]{Diagrams/vector4.png}\\
\hline
	Opposite Direction 					& $(a \cdot b) < 0$ &\includegraphics[scale=0.1]{Diagrams/vector5.png}\\
\hline

\end{tabular}
\caption{Table showing the dot product tests and an example of their use.}
\label{dot product test}
\end{table}
\FloatBarrier

\noindent
The cross product, also known as the outer product, takes two different vectors and finds the perpendicular vector of those two vectors. The cross product is only possible in 3D space and can be expressed in the following formula using the left-hand rule.

\begin{equation}
a \times b = [(a_y b_z - a_z b_y), (a_z b_x - a_x b_z), (a_x b_y - a_y b_x)]
\end{equation}

\noindent
The result of a cross product can be seen in figure \ref{Cross product diagram} below. Where vectors $a$ and $b$ give the perpendicular vector $a \times b$. The cross product is beneficial within physics calculations when it's necessary to find the rotational motion of objects. It is also used in the graphics shader when finding the normal vector in light calculations.


\begin{figure}[htbp]
	{\centering
		\setlength{\fboxrule}{1pt}
		\vspace{7px}
		\fbox{
			\includegraphics[scale=0.25]{Diagrams/cross_product.png}
			\label{Cross product diagram}
		}
		\caption{Diagram of the cross product of two vectors a and b.}
	}
\end{figure}
\FloatBarrier

\noindent
Some properties of the cross product are as follows:

\begin{itemize}
	\item It is non-commutative, meaning order matters($a \times b \not= b \times a$).
	\item It is anti-commutative ($a \times b = -(a \times b)$).
	\item It is distributive with addition ($a \times (b + c) = (a \times b) + (a \times c)$).
\end{itemize}

\section{Matrices}

A model in 3D space exists as a set of position vertices, often represented as vectors. Moving the model requires moving all of the vertices of that model without distorting it in any way. Moving a model like this is called a model transform. There are four main types of transforms, these being: translation, rotation, scale, and shear. Matrices are a single mathematical construct capable of carrying out all four of these transformations. This section will only cover the first three as the shear transformation only used in certain circumstances and will not be useful in this thesis.    

A matrix is a 2D array of numbers arranged into rows and columns, which can come in many different sizes. In 3D graphics, matrices used for transformations are 3 $\times$ 3 and 4 $\times$ 4 matrix, as seen below. 

\begin{equation}
\textbf{M} = \begin{bmatrix}
M_{11} & M_{12} & M_{13} \\
M_{21} & M_{22} & M_{23} \\
M_{31} & M_{32} & M_{33}
\end{bmatrix}
\end{equation}

\begin{equation}
\textbf{M} = \begin{bmatrix}
M_{11} & M_{12} & M_{13} & M_{14}\\
M_{21} & M_{22} & M_{23} & M_{24}\\
M_{31} & M_{32} & M_{33} & M_{34}\\
M_{41} & M_{42} & M_{43} & M_{44}
\end{bmatrix}
\end{equation}

\noindent
A 3 $\times$ 3 matrix is used for linear transforms such as scaling and rotation. Furthermore, a linear transform that contains translation is known as an affine transform and is represented as a 4 $\times$ 4 matrix known as an Atomic Transform Matrix. An atomic Transform matrix is the concatenation of four 4 $\times$ 4 matrices, one for translation, rotation, scale, and shear transforms, which results in a 4 $\times$ 4 matrix, as shown below. 

The affine matrix can be shown in the expression below where $RS$ is a 3 $\times$ 3 matrix containing the rotation and scale where the $4^th$ elements are 0. The $T$ elements represent the translation, with the 4th element being 1. 

\begin{equation}
\textbf{M} = \begin{bmatrix}
RS_{11} & RS_{12} & RS_{13} & 0\\
RS_{21} & RS_{22} & RS_{23} & 0\\
RS_{31} & RS_{32} & RS_{33} & 0\\
T_{1} & T_{2} & T_{3} & 1
\end{bmatrix}
\end{equation}

\noindent
The product of two linear transform matrices will be another linear transform matrix where both of the transformations have taken place. This is true for the multiplication of two affine transform matrices as well, and is why matrix multiplication is so powerful in 3D graphics. Take the two matrices, $A$ and $B$, which give the product $P$. To multiply $A$ and $B$, the dot product of the row and the column is must be calculated, which can be seen in the equation below. It is also important to know that matrix multiplication is non-commutative $(AB \not= BA)$.

\begin{equation}
\textbf{AB} = \begin{bmatrix}
A_{11} & A_{12} & A_{13}\\
A_{21} & A_{22} & A_{23}\\
A_{31} & A_{32} & A_{33}
\end{bmatrix}
\times
\begin{bmatrix}
B_{11} & B_{12} & B_{13}\\
B_{21} & B_{22} & B_{23}\\
B_{31} & B_{32} & B_{33}
\end{bmatrix}
= \begin{bmatrix}
(A_{row1} \cdot B_{col1}) & (A_{row1} \cdot B_{col2}) & (A_{row1} \cdot B_{col3})\\
(A_{row2} \cdot B_{col1}) & (A_{row2} \cdot B_{col2}) & (A_{row2} \cdot B_{col3})\\
(A_{row3} \cdot B_{col1}) & (A_{row3} \cdot B_{col2}) & (A_{row3} \cdot B_{col3})
\end{bmatrix}
\end{equation}

\noindent
Translating a vertex in 3D space using matrices is is relatively straightforward. The vertex can be is added to the matrix as seen in the equation below. Where $V$ is the vertex and the $T$ is the translation matrix. To rotate an entire model, the same translation matrix can be applied to all vertices.

\begin{equation}
V + T = \begin{bmatrix}
V_{x} \\
V_{y} \\
V_{z}~ \\
1
\end{bmatrix}
+
\begin{bmatrix}
1 & 0 & 0 & 0\\
0 & 1 & 0 & 0\\
0 & 0 & 1 & 0\\
T_{x} & T_{y} & T_{z} & 1
\end{bmatrix}
= \begin{bmatrix}
(V_x + T_x)~ \\
(V_y + T_y)~ \\
(V_z + T_z)~ \\
1
\end{bmatrix}
\end{equation}

\noindent
To rotate a vertex in 3D space, the vertex position and rotation angle can be applied to the matrix differently depending on the axis about which it is rotating. Similar to the translation matrix, rotation matrices are applied to each vertex, to gain the new position of the vertex. The rotation matrices below rotate a vertex around the x, y, and z axes, repectively. 

\begin{equation}
R_x(\theta) = 
\begin{bmatrix}
(v_x)~ \\
(v_y)~ \\ 
(v_z)~ \\
1
\end{bmatrix}
\begin{bmatrix}
1 	& 0 					& 0 					& 0\\
0 	& \text{cos}(\theta) 	& \text{sin}(\theta) 	& 0\\
0 	& -\text{sin}(\theta) 	& \text{cos}(\theta) 	& 0\\
0 	& 0 					& 0 					& 1
\end{bmatrix}
\end{equation}

\begin{equation}
R_y(\theta) = 
\begin{bmatrix}
(v_x)~ \\
(v_y)~ \\
(v_z)~ \\
1
\end{bmatrix}
\begin{bmatrix}
\text{cos}(\theta) 	& 0 					& -\text{sin}(\theta) 	& 0\\
0 					& 1						& 0						& 0\\
\text{sin}(\theta) 	& 0 					& \text{cos}(\theta)	& 0\\
0 					& 0 					& 0 					& 1
\end{bmatrix}
\end{equation}

\begin{equation}
R_z(\theta) = 
\begin{bmatrix}
(v_x)~ \\
(v_y)~ \\
(v_z)~ \\
1
\end{bmatrix}
\begin{bmatrix}
\text{cos}(\theta) 	& \text{sin}(\theta) 	& 0						& 0\\
-\text{sin}(\theta) & \text{cos}(\theta) 	& 0
					& 0\\
0 					& 0 					& 1						& 0\\
0 					& 0 					& 0 					& 1
\end{bmatrix}
\end{equation}

\noindent
Similarly, the scale transform takes a vertex and multiplies it by the scale matrix. If there are a large number of vertices making up an entire model if all of the points are scaled using the scale transform, the result will be the model either increasing or decreasing in size.

\begin{equation}
VS = \begin{bmatrix}
V_{x} \\
V_{y} \\
V_{z} \\
1
\end{bmatrix}
\begin{bmatrix}
S_x & 0 & 0 & 0\\
0 & S_y & 0 & 0\\
0 & 0 & S_z & 0\\
0  & 0  & 0 & 1
\end{bmatrix}
= \begin{bmatrix}
(V_x S_x)~ \\
(V_y S_y)~ \\
(V_z S_z)~ \\
1
\end{bmatrix}
\end{equation}

\section{Quaternions}

In computer graphics, there are several ways to represent 3D rotations. One method is to use matrix affine transforms, that is spoken about in the previous section. Matrices are a common way of representing rotation; however, there are some limitations. Matrices are represented by nine floating-point values and can be computationally expensive to store and process, mainly when doing a vector to matrix multiplication. There are also situations where it is necessary to interpolate from one rotation to another, smoothly, or to find the rotation somewhere between two different rotations. It is possible to make these calculations using matrices, but it can become very complicated and even more computationally expensive. Quaternions are the answer to these challenges.

Quaternions look similar to a 4D vector. They contain four axes $q = [q_x, q_y, q_z, q_w]$, these are represented with a real axis ($q_w$) and three imaginary axes ($q_x, q_y, q_z$). A quaternion can be represented in the complex form below: 

\begin{equation}
q = (iq_x + jq_y + kq_z + qw)
\end{equation}

\noindent
For this thesis, it is not essential to understand the derivation of quaternions mathematically. However, it is essential to understand that a quaternion obeying the rule in \ref{unit quat} below is known as a unit quaternion.

\begin{equation} \label{unit quat}
	q_x^2 + q_y^2 + q_z^2 + q_w^2 = 1
\end{equation}

\noindent
Unit quaternions can be used for rotations, and it is possible to convert a quaternion to a unit quaternion by taking the angle and the axis of rotation and applying it to the quaternion as seen in \ref{unit quat conversion} below.

\begin{equation} \label{unit quat conversion}
\begin{aligned}
& q = [q_x, q_y, q_z, q_w]\\
& \text{where} \\
& q_x = a_x sin \frac{\theta}{2}\\
& q_y = a_y sin \frac{\theta}{2}\\
& q_z = a_z sin \frac{\theta}{2}\\
& q_w = cos \frac{\theta}{2}
\end{aligned}
\end{equation}

\noindent
The scalar part ($q_w$) is the cosine of the half-angle, and the vector part ($q_x q_y q_z$) is the axis of the rotation, scaled by the sine of the half-angle of rotation. 

Some of the most useful features of a quaternion are the ability to rotate vectors, interpolate between two rotations, and concatenate rotations together.

The first operation for quaternions is addition. The addition of two quaternions is quite simple. It involves taking each component of each quaternion and adding them together. This method is similar to matrix addition and can be expressed as follows.

\begin{equation}
p + q = [(p_w + q_w), (p_x + q_x), (p_y + q_y), (p_z + q_z)]
\end{equation}

\noindent
The multiplication of quaternions is also incredibly powerful and can be used to concatenate many rotations together without any gimbol lock. There are several different types of quaternion multiplication. However, the one most commonly used for quaternion rotation is called the Grassmann product. 

The Grassmann product can be expressed in the formula below. Where $p$ and $q$ are quaternions, and the subscript $w$ indicates the scalar part and subscript $x, y, z$ indicates the vector components of each quaternion.


\begin{equation}
\begin{aligned}
& R = r_w + r_x + r_y + r_z\\
& \text{where}\\
& r_w = p_w q_w - (p_x q_x + p_y q_y + p_z q_z)\\
& r_x = p_w q_x + p_x q_w + p_y q_z - p_z q_y\\
& r_y = p_w q_y + p_y q_w - p_x q_z + p_z q_x\\
& r_z = p_w q_z + p_z q_w + p_x q_y - p_y q_x\\
\end{aligned}
\end{equation}

\noindent
Multiplying two quaternions together is important for multiple rotations taking place one after the other. However, rotate a quaternion by a vector, the vector will need to be converted into its quaternion form. This requires taking the unit vector $v$ and using it as the vector part of the quaternion with a scalar part being equal to zero. This can be written as $Q_v = [v, 0] = [v_x, v_y, v_z, 0]$. The Grassmann product can be used to apply the rotation, by taking the product of the rotation quaternion $q$ and the vector form quaternion $v$ and the inverse of the rotation quaternion $q^-1$.

\begin{equation}
	V_q = qvq^{-1}
\end{equation}

\noindent
The conjugate and the inverse of a unit quaternion are identical. The conjugate or inverse of a unit quaternion can be calculated by negating the vector components `$q_v$' of the quaternion while leaving the scalar component `$q_s$' the same. The inverse of a unit quaternion can be expressed as follows.

\begin{equation}
	q^{-1} = [-q_v, q_s]
\end{equation}

\noindent
Concatenating quaternion rotations together is similar to how matrix affine transformations can be multiplied together. The Grassman product can be used. The Grassman product is noncommutative and, therefore, order matters. The quaternion multiplication would result in the rotation that represents all rotations, if they were to happen one after the other. This can be expressed in the equation below.

\begin{equation}
\begin{aligned}
Q_{net} = Q_3 Q_2 Q_1\\
\end{aligned}
\end{equation}

\noindent
The order the quaternions $Q_1, Q_2$, and $Q_3$ is applied is: $Q_3$, $Q_2$, and then $Q_1$. The product of three quaternions can be applied to a vector by multiplying the product of the quaternions to the vector, then multiplying the product of the inverse of the quaternion as seen below. 

\begin{equation}
\begin{aligned}
v' = Q_3 Q_2 Q_1~ v~ Q_{1}^{-1} Q_{2}^{-1} Q_{3}^{-1}
\end{aligned}
\end{equation}

\noindent
Another incredibly useful mathematical function is called rotational linear interpolation, also known as \acrshort{lerp}. The \acrshort{lerp} function takes two quaternions, $Q_1$ and $Q_2$, and linearly interpolates between those two rotations by a given percentage  $\beta$. The \acrshort{lerp} function can be defined as follows.

\begin{equation}
\begin{aligned}
& Q_{\text{LERP}} = \text{LERP}(Q_1, Q_2, \beta) = \frac{(1-\beta)Q_1 + \beta Q_2}{\mid(1-\beta)Q_1 + \beta Q_2\mid} \\
& = \text{normalize} \left(\begin{bmatrix}
(1 - \beta) Q_{1x} + \beta Q_{2x}\\
(1 - \beta) Q_{1y} + \beta Q_{2y}\\
(1 - \beta) Q_{1z} + \beta Q_{2z}\\
(1 - \beta) Q_{1w} + \beta Q_{2w}					
\end{bmatrix}\right)
\end{aligned}
\end{equation}

\noindent
Using the linear interpolation function will result in a rotation between $Q_1$ and $Q_2$ at a given percentage of $\beta$, which can be specified between 0 and 1. Where 0 is the rotation of $Q_1$ and 1 is rotation of $Q_2$. \acrshort{lerp} is very helpful in many areas of 3D graphics and is used extensively within the physics simulation of branches covered in chapter \ref{physics chapter}.

\section{summary}

This chapter covers the three major mathematical concepts used for representing a 3D objects' position, rotation, and scale within 3D graphics applications. This includes moving objects around a scene, or for animation or simulation of an object. It is essential to understand these concepts when implementing the L-system interpreter, as it is used to manipulate the branches or objects. These concepts are also useful in the implementation of the physics simulations. The \acrlong{glm} (\acrshort{glm}) library provides a large number of useful classes and functions for working with vertices, matrices, and quaternions and can be used instead of re-implementing these mathematical functions.






