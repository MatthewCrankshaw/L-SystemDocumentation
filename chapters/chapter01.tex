\begin{flushleft}

This thesis explores the procedural generation and simulation of three dimensional plants and trees using a Lindenmayer System (L-system), as well as exploring the advantages and disadvantages of different types of L-systems. Procedural generation in the context of plants and trees can be thought of as taking a template or description, that describes the necessary information to generate structures and data algorithmically, which can be used to render a model within a three dimensional application. The procedural generation should also be capable of providing the relavent data to simulate the physical properties of the rendered model in a realistic way, for example the simulating gravity, wind or even rain. \\

Plants and trees can have seemingly complex and random structures, however, with closer observation, trees of a similar species have very obvious traits and features, for instance a palm tree or Arecaceae, have long stright trunks with leaves exclusively near the top, the leaves are long, compound leaves, branching in all different directions. Comparatively a pine tree has a long staight trunk with many branches coming off in different directions nearly pupendicular to the ground, all the way from its base to the top of the trunk. These are two very different species of trees, however they share very similar properties. 

In order for procedural generation to be useful it must be relatively straightforward to define a description of a certain type of tree, in an way that accuratly represents it, furthermore, the description must be able to fit many different species of trees with varying charactoristics, and does not neccessarily have to be limited to known species of trees as some graphics application may want something that is other-wordly.




\end{flushleft}

\section{Motivations}
 
\begin{flushleft}
One of the most time consuming parts for digital artists and animators is creating differing variations of the same basic piece of artwork. In most games and other graphics applications environment assets such as trees, plants, grass, algae and other types of plant life make up the large majority of the assets within a game. Creating a tree asset can take a skilled digital artist more than an hour of work by hand, The artist will then have to create many variations of the same asset in order to obtain enough variation that a user of that graphics application would not notice that the asset has been duplicated. If you multiply this by the number of assets that a given artist will have to create and then modify, you are looking at an incredible number of hours that could potentially be put to use creating much more intricate and important assets.\\
\vspace{5mm}
In addition to the huge number of development hours required, it is also important to note that graphics assets are then stored in large data files, describing the geometry and textures and other information. If we require three very similar plants, we have to store three separate sets of data. Procedurally generating plants can avoid this wasteful data storage entirely. We could just store one specification or description of set of similar plants we would like to create, then procedurally generate the geometry during the running of the program.\\
\end{flushleft}

\section{Research Aims and Objectives}

\begin{flushleft}

To develop upon the Lindenmayer System in order to procedurally generating the structure of plant life in real time, in a way that allows us to specify the species, or overall look of the plant as well as introduce variation in order to produce plants that look similar, but can vary in shape, size and branching structure. \\

\vspace{5mm}

I will also investigate using the Lindenmayer System to specify aspects of the plant life that enables the simulation of physical behaviour such as external forces like gravity and wind, and thus having a specification not only for the development of the plants structure but also of its physical behaviour. \\

\vspace{5mm}

Finally, I will be investigating a method of generating a 3D mesh for the given structure of plant, where the branches are seamlessly connected together, and textures are intelligently mapped onto the generated 3D mesh.\\ 

\end{flushleft} 


\section{Scope and Limitations}

\begin{flushleft}

For the purpose of this thesis, we will only be focusing on larger plant life, such as flowers, bushes and trees. We will not be focusing on algae or fungi as these types of plants are usually better represented with specialised texturing in modern 3D applications.\\

\vspace{5mm}



\end{flushleft}

\section{Timeframe}

\begin{flushleft}

This research will be carried out over the period of a full year, from the 20\textsuperscript{th} of February 2019 through to the 20\textsuperscript{th} of February 2020. 

\end{flushleft}

\section{Structure of Thesis}




