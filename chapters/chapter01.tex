\begin{flushleft}

Procedurally generating three dimensional models of plant-life is a difficult task, largely due to the complex branching structures and how different species of plants can vary so wildly in structure and detail. It has only been in the last few decades, in the boom of computer graphics that research has been conducted into how to best generate three dimensional models of plant-life. There are three main techniques for generating plant models, these include extracting and reconstructing real world data, using modeling software or using a procedural or rule based generator. This thesis explores the procedural generation and simulation of three dimensional trees using a Lindenmayer System (L-system), as well as exploring the advantages and disadvantages of different types of L-systems.

\end{flushleft}

\section{Introduction to Procedural Generation}

\begin{flushleft}

Procedural generation is used in many different areas and applications particularly in computer graphics, particularly when generating naturally occuring structures such as plants or terrain. There is an aspect of randomness that is implicit when talking about procedural generation. Generated structures should fit a certain description that is given to it, but it must also provide some form of variation such that two instances of the same description will almost never be identical. There are three main methods for procedurally generating models of trees, these are genetic algorithms \cite{haubenwallner2017shapegenetics}, space colonisation algorithms\cite{juuso2017procedural} and the Lindenmayer system. Both genetic algorithm and space colonisation algorithms are similar in that they require a description in 3D space of which describes what the trees general size and shape should look like, the algorithm is then responsible for creating branches and matching it to that original template. L-systems on the other hand operate quite differently, the L-system defines a formal grammar, which contains a set of sybols that belong to an alphabet. A set of symbols is then chosen as the starting point and a set of production rules is used to dictate which symbols can be rewriten and what they will be rewritten with. A full explanation of L-system will be spoken about in chapter \label{l-system chapter}. In essence, the L-system uses the set of production rules to generate a structure that follows those rules. This requires specifying a template or description of a plant or tree, that describes the necessary information to generate the overall structure and data algorithmically. This structure and data can then be used to render a model within a three dimensional application. Trees can have complex and random structures, however, with closer observation, trees of a similar species have very obvious traits and features, for instance a palm tree (Arecaceae) has long stright trunks with leaves exclusively near the top, the leaves are long, compound leaves, branching in all different directions. Comparatively a pine tree has a long staight trunk with many branches coming off in different directions pupendicular to the ground, from its base to the top of the trunk. These are two very different species of trees and look quite different, however they share very similar properties. The challenge behind procedurally generating and simulating trees is how to provide a human readable grammar that describes in sufficient detail, how it should generate and render the three dimensional model, whilst allowing for randomness and variety within the generation process. It must be relatively straightforward and intuitive to define the procedural generation description, and must accuratly represents it, furthermore, the description must be able to fit many different species of trees with varying charactoristics, and must not be limited to only known species of trees, as some graphics applications may require something that is other-wordly.

\end{flushleft}

\section{Introduction to Rewriting Systems}

\begin{flushleft}

String rewriting also known as rewrite systems are the fundimental concept behind L-systems. In their most basic form rewrite systems are a set of symbols or states and a set of relations that dictate how they transform from one state to the other. Rewrite systems can be non-deterministic, meaning that there could be a transition that depends on a condition being met or neighbouring states to be of a certain type. Using this rewriting concept any preceeding state can rely upon the current state as well as any conditions neccessary for transformation, otherwise the state will remain the same until such a point that the condition is satisfied. 

\end{flushleft}

\section{Motivations}
 
\begin{flushleft}
One of the most time consuming parts for digital artists and animators is creating differing variations of the same basic piece of artwork. In most games and other graphics applications environment assets such as trees, plants, grass, algae and other types of plant life make up the large majority of the assets within a game. Creating a tree asset can take a skilled digital artist more than an hour of work by hand, The artist will then have to create many variations of the same asset in order to obtain enough variation that a user of that graphics application would not notice that the asset has been duplicated. If you multiply this by the number of assets that a given artist will have to create and then modify, you are looking at an incredible number of hours that could potentially be put to use creating much more intricate and important assets.\\
\vspace{5mm}
In addition to the huge number of development hours required, it is also important to note that graphics assets are then stored in large data files, describing the geometry and textures and other information. If we require three very similar plants, we have to store three separate sets of data. Procedurally generating plants can avoid this wasteful data storage entirely. We could just store one specification or description of set of similar plants we would like to create, then procedurally generate the geometry during the running of the program.\\
\end{flushleft}

\section{Research Aims and Objectives}

\begin{flushleft}

To develop upon the Lindenmayer System in order to procedurally generating the structure of plant life in real time, in a way that allows us to specify the species, or overall look of the plant as well as introduce variation in order to produce plants that look similar, but can vary in shape, size and branching structure. \\

\vspace{5mm}

I will also investigate using the Lindenmayer System to specify aspects of the plant life that enables the simulation of physical behaviour such as external forces like gravity and wind, and thus having a specification not only for the development of the plants structure but also of its physical behaviour. \\

\vspace{5mm}

Finally, I will be investigating a method of generating a 3D mesh for the given structure of plant, where the branches are seamlessly connected together, and textures are intelligently mapped onto the generated 3D mesh.\\ 

\end{flushleft} 


\section{Scope and Limitations}

\begin{flushleft}

For the purpose of this thesis, we will only be focusing on larger plant life, such as flowers, bushes and trees. We will not be focusing on algae or fungi as these types of plants are usually better represented with specialised texturing in modern 3D applications.\\

\vspace{5mm}



\end{flushleft}

\section{Timeframe}

\begin{flushleft}

This research will be carried out over the period of a full year, from the 20\textsuperscript{th} of February 2019 through to the 20\textsuperscript{th} of February 2020. 

\end{flushleft}

\section{Structure of Thesis}




