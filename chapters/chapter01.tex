
\lettrine[lines=3]{P}{}rocedurally generating 3D models of plant-life is a challenging task, largely due to the complex branching structures and variation between different types of plant species. Up until recently all assets within 3D graphics applications either had to be scalpted by hand using a 3D modeling software, or scanned using photogrammetry, laser triangulation or some form of contact based 3D scanning. These methods are still used today but tend to be very time consuming and extremely costly. With the increase in computational power over the last few decades more emphasis has been placed on the use of produral generation to create complex structures such as tarrain, architecture, sound, characters and weaponry with far greater speed than previous techniques and much better realism than would be possible with artists, however, plant-life still stands as a challenge as it is difficult to define a system that is capable of representing every possibly type of plant-life, whilst having a simple description for what the procedural generator should output. The Lindenmayer System (L-system) stands as a solution to this problem, it was originally developed by Aristed Lindenmayer in order to represent the development of multicellular organisms \cite{lindenmayer1968mathematical}, but has since gained popularity in the area of procedural generation and has been adapted to represent different types of structures, mainly with plant-life, such as trees, flowers, algea and grasses, but the L-system is also applicable to non-organic structures such as music, artificial neural networks and tiling patterns \cite{Prusinkiewicz1989}. The L-system in its most basic form is a formal grammar which contains a set of alphanumeric characters and symbols that belong to an \textit{alphabet}, the alphabet is used to create a starting string and a set of production rules. The production rules are applied to the starting string to dictate which symbols can be rewriten and what they will be rewritten with. In essence, the L-system uses the set of production rules for string rewriting in order to generate a string of symbols that follows those production rules. The resuting string is then interpreted in a way that best fits its representation, in this case to produce a model of a plant. \\
This thesis develops upon the L-system in order to procedurally generate structures of plant-life in real-time, the L-system grammar allows the specification of the structure of a plant to be described in a human readable, formal grammar, and to specify variation in shape, size and branching structure within a particular species. Furthermore, this thesis will also investigate the use of a parameterised L-systems to propegate physical properties using string rewriting which will enable the animation of the physical behaviour, thus making it possible to simulate external forces such as gravity and wind.




\section{Introduction to Procedural Generation}

\begin{flushleft}

Procedural generation is used in many different areas and applications particularly in computer graphics, particularly when generating naturally occuring structures such as plants or terrain. There is an aspect of randomness that is implicit when talking about procedural generation. Generated structures should fit a certain description that is given to it, but it must also provide some form of variation such that two instances of the same description will almost never be identical. There are three main methods for procedurally generating models of trees, these are genetic algorithms \cite{haubenwallner2017shapegenetics}, space colonisation algorithms\cite{juuso2017procedural} and the Lindenmayer system. Both genetic algorithm and space colonisation algorithms are similar in that they require a description in 3D space of which describes what the trees general size and shape should look like, the algorithm is then responsible for creating branches and matching it to that original template. L-systems on the other hand operate quite differently,  This structure and data can then be used to render a model within a three dimensional application. Trees can have complex and random structures, however, with closer observation, trees of a similar species have very obvious traits and features, for instance a palm tree (Arecaceae) has long stright trunks with leaves exclusively near the top, the leaves are long, compound leaves, branching in all different directions. Comparatively a pine tree has a long staight trunk with many branches coming off in different directions pupendicular to the ground, from its base to the top of the trunk. These are two very different species of trees and look quite different, however they share very similar properties. The challenge behind procedurally generating and simulating trees is how to provide a human readable grammar that describes in sufficient detail, how it should generate and render the three dimensional model, whilst allowing for randomness and variety within the generation process. It must be relatively straightforward and intuitive to define the procedural generation description, and must accuratly represents it, furthermore, the description must be able to fit many different species of trees with varying charactoristics, and must not be limited to only known species of trees, as some graphics applications may require something that is other-wordly.

\end{flushleft}

\section{Introduction to Rewriting Systems}

\begin{flushleft}

String rewriting also known as rewrite systems are the fundimental concept behind L-systems. In their most basic form rewrite systems are a set of symbols or states, and a set of relations or production rules that dictate how they transform from one state to the other \cite{prusinkiewicz2012algorithmic}. In this way we are able to generate complex structures by successively replacing parts of a initial simple object with more complex parts. Rewrite systems can be non-deterministic, meaning that there could be a transition that depends on a condition being met or on a neighbouring states to be of a certain type. Using this rewriting concept any preceeding state can rely upon the current state as well as any conditions neccessary for transformation, if neither of these are met the state will remain the same, and will be checked in the next rewriting stage. A graphical representation of an object defined in rewriting rules can be seen below in figure \ref{snowflake curve} below, called the snowflake curve proposed by Von Koch \cite{koch1906methode}.

\begin{figure}[htbp]
	{\centering
		\setlength{\fboxrule}{1pt}
		\vspace{7px}
		\fbox{
			\includegraphics[scale=0.3]{Diagrams/snowflakeCurve.png}
		}
		\caption{Construction of the snowflake curve\cite{prusinkiewicz2013lindenmayer}.} \label{snowflake curve}
	}
\end{figure}
\FloatBarrier

The snowflake curve shown in figure \ref{snowflake curve} above, starts with two parts the initiator which is the initial set of edges forming a certain shape, and the generator which is a set of edges. The generator replaces each edge of the initiator forming a new shape, that new shape then becomes the new initiator where each edge is again replaced by the generator, and so on. The result is a complex shape similar to that of a snowflake. The initiator, generator concept can be adapted and represented as a set of strings capable of producing a similar result.

\end{flushleft}

\section{Introduction to Grammars}

\begin{flushleft}

In the context of computer science, grammars are defined as a set of rules governing which strings are valid or allowable in a language or text. They consist of syntax, morphology and semantics. Formal languages have been defined in the form of grammars to suit particular problem domains. It is natural for Humans to communicate a solution in the form of language, it is therefore intuitive to use a language to describe the desired outcome when dealing with the procedural generation of plant-life. In the past grammars have been used extensively in computer science in the form of programming languages which provide a computer with a set of instructions to carry out to gain an expected result. The challenge is therefore to create a grammar in the form of a rewriting system that facilitates the procedural generation of plant-life. A rewriting system such as the L-system operates in a way that is consistant with a context-free class of Chomsky grammar \cite{chomsky1956three}, similar to that of the programming language ALGOL-60 introduced by Backus and Naur in  1960\cite{backus1960report}. In figure \ref{chomsky grammars} below, there are two types of L-system grammars that overlap the classes of chomsky grammars, the OL-system and the 1L-system. The details of these two systems will be discussed in detail chapter \ref{l-system chapter}. 0L-systems are grammars that can represent a context-sensitive Chomsky grammar but generally tend to be context-free, the main difference between the 0L-system and the 1L-system is that 1L-systems can be recursively enumerable. Furthermore, it is possible for a 1L-system to represent any 0L-system, therefore, 1L-system languages tend to be more complex and verbose when compared to 0L-systems, this creates a trade off between a more powerful and complex language or a less powerful but simpler language. \\

\begin{figure}[htbp]
	{\centering
		\setlength{\fboxrule}{1pt}
		\vspace{7px}
		\fbox{
			\includegraphics[scale=0.5]{Diagrams/ChomskyGrammar.png}
		}
		\caption{Chomsky classes of grammars with relation to the 0L and 1L systems generated by L-systems.} \label{chomsky grammars}
	}
\end{figure}
\FloatBarrier

\end{flushleft}

\section{Motivations}
 
\begin{flushleft}

One of the most time consuming parts for digital artists and animators is creating differing variations of the same piece of artwork. In most games and other graphics applications environment assets such as trees, plants, grass, algae and other types of plant life make up the large majority of the assets within a game. Creating a tree asset can take a skilled digital artist more than an hour of work by hand, The artist will then have to create many variations of the same asset in order to obtain enough variation that a user of that graphics application would not notice that the asset has been duplicated. If you multiply this by the number of assets that a given artist will have to create and then modify, you are looking at an incredible number of hours that could potentially be put to use creating much more intricate and important assets. In addition to this, it is also important to note that graphics assets are then stored in large data files, describing the geometry and textures and other information. If we require three very similar plants, we have to store three separate sets of data. Procedurally generating plants can avoid this wasteful data storage entirely. We could just store one specification or description of set of similar plants we would like to create, then procedurally generate the geometry during the running of the program. \\

\vspace{5mm} 

The L-system can also not only procedurally generate the geometry of the plant-life but can also generate parameters physical properties of the plant inself such as the weight and flexability of branches as well as its wind resistance and many other important information that can be used to simulate or animate the motion of the plant under various forces. \\

\end{flushleft}

\section{Structure of Thesis}

\begin{flushleft}

This thesis is split into three major parts. Part 1 focuses on the L-system itself, it defines the various types of L-systems for modeling plant-life, the concept of a parametric L-system as well as some techniques for definiting randomness and stochasticisity within an L-system in order to create variaty. Part 2 talks about the L-system rewriter, this that and how it is implemented in order to generate the structures which will be rendered in the final part. Part 3 focuses on the interpretation of the L-system rewriters result and finally how the L-system is used to finally render a convincing model of a tree on the screen, as well as how the L-system is capable of providing information relavant to the simulation or animation of the generated plants.

\end{flushleft}




