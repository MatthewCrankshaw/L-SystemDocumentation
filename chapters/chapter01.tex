\section{Context and Background}
One of the most time consuming parts for digital artists and animators is creating differing variations of the same basic piece of artwork. In most games and other graphics applications environment assets such as trees, plants, grass, algae and other types of plant life make up the large majority of the assets within a game. Creating a tree asset can take a skilled digital artist more than an hour of work by hand, The artist will then have to create many variations of the same asset in order to obtain enough variation that a user of that graphics application would not notice that the asset has been duplicated. If you multiply this by the number of assets that a given artist will have to create and then modify, you are looking at an incredible number of hours that could potentially be put to use creating much more intricate and important assets.

The unique thing about plant life when compared with other types of graphics assets is that plant life is very random in the way it grows and it does not take an incredibly realistic model of a plant life to trick the human mind into believing that what it is seeing is some kind of plant. However what stands out like a sore thumb is when plants that are growing next to each other seemingly have no influence on the plant life around them.
