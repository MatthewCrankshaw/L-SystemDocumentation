\begin{flushleft}

To procedurally generate realistic plant like structures in a way that can be used for modern graphics applications, as well as simulate outside forces such as gravity and wind on the generated structure in real time.

\end{flushleft}

\section{Motivations}
 
\begin{flushleft}
One of the most time consuming parts for digital artists and animators is creating differing variations of the same basic piece of artwork. In most games and other graphics applications environment assets such as trees, plants, grass, algae and other types of plant life make up the large majority of the assets within a game. Creating a tree asset can take a skilled digital artist more than an hour of work by hand, The artist will then have to create many variations of the same asset in order to obtain enough variation that a user of that graphics application would not notice that the asset has been duplicated. If you multiply this by the number of assets that a given artist will have to create and then modify, you are looking at an incredible number of hours that could potentially be put to use creating much more intricate and important assets.\\
\vspace{5mm}
In addition to the huge number of development hours required, it is also important to note that graphics assets are then stored in large data files, describing the geometry and textures and other information. If we require three very similar plants, we have to store three separate sets of data. Procedurally generating plants can avoid this wasteful data storage entirely. We could just store one specification or description of set of similar plants we would like to create, then procedurally generate the geometry during the running of the program.\\
\end{flushleft}

\section{Research Aims and Objectives}

\begin{flushleft}

To develop a generalised method of procedurally generating the structure of plant life, while being able to specify a species, or overall look of the plant as well as introduce variation in order to produce plants that look similar, but vary in shape, size and branching structure. \\
I will also investigate a way of simulating the physical behaviour of the plant when external forces are applied such as gravity and wind. \\
Finally I will be investigating a method of generating a 3D mesh for a given structure of plant, where the branches are seemlesly connected together and textures are intelligently mapped onto the procedurally generated 3D mesh.\\ 
\end{flushleft} 


\section{Scope and Limitations}

\section{Timeframe}



\section{Structure of Thesis}


