
Producing and simulation of realistic looking plant-life assets in 3D applications is a challenging task. Research has shown that the L-systems' rewriting mechanism is an effective means for procedurally generating structures that represent multiple types of plant-life, those structures can be interpreted as a set of instructions to render a realistic model. The relationship between the rewriting mechanism of the L-system and the way that the resulting structure is interpreted and rendered on the screen, and to determine if an L-system can manipulate the physical properties of branches to provide the relevant information needed to physically simulate a plants reaction to wind and gravity. 

A parametric L-system was developed, that can create plant structures that have variation in their branching structure and physical features, and that are capable of holding the parameters for a physical simulation. A system for representing and simulating the result generated by the rewriting system was implemented and tested on L-systems of varying complexity. The results show that an L-system can be created not only to represent a plants structure but also its physical properties, with the dissadvantage that the L-system becomes more cumbersome and difficult to understand. On this basis, it is beneficial to have a plant assets that not only are procedurally generated but also can be easily simulated.