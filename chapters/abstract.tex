Producing and simulating realistic-looking plant-life assets for 3D applications is a challenging task. An important contributing factor in the realism of plant models in modern graphics applications is its motion, but creating plant assets that both look and move realistically is a tedious and time-consuming process. Lindenmayer systems are a useful tool for producing a set of instructions that represent the structures of organic life, such as algae, flora, and trees. These instructions can be interpreted using turtle graphics to render realistic models. A class of L-system known as parametric L-systems can provide extra information through the rewriting process using parameters. The use of parametric L-systems is investigated to provide both the physical and geometric properties of a plant, such that a model can be rendered and physically simulate the effects of gravity and wind. The relationship between the  L-systems' rewriting mechanism and the interpreter system is investigated and discussed. 

The parametric class of L-system is a grammar similar to that of a recursive programming language. A compiler-like software solution is developed, that is capable of taking L-system language as input and producing instructions and information to the interpreter system. A three-stage 3D graphics software system is implemented to interpret the L-system instructions and information in order to display complex plant models. A separate physics system is also developed to simulate the motion of the resulting plant models under gravity or wind.

There is a trade-off between the complexity of the rewriting system and the interpreting system. Consideration as to the advantages and disadvantages of these trade-offs is discussed. It is shown that parametric L-systems can create plant structures that have variations in their branching structure and physical features, which can provide the physical properties of branches necessary to simulate forces like gravity and wind. There is considerable benefit to having a software system produce both the geometry of a plant model and the information necessary for simulation, as it allows a plant to be defined in a single definition in the form of an L-system.
