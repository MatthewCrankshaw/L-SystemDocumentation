
The procedural generation and simulation of realistic looking plant-life assets in 3D applications is a challenging task. Research has shown that the L-systems' rewriting mechanism is an effective means to procedurally generate structures that represent plant-life, those structures can be further interpreted to render a realistic model of that structure. This study aims to investigate the relationship between the rewriting mechanism of the L-system and the way that the resulting structure is interpreted and rendered on the screen, as well as detarine if an L-system can manipulate the physical properties of branches to provide the relevant information needed to physically simulate a plants reaction to wind and gravity. 

After a review of the literature, a suitable rewriting system was developed, which is capable of holding the parameters for a physical simulation. A system for representing and simulating the result generated by the rewriting system was implemented and tested on L-systems of varying complexity. The results show that an L-system can be created not only to represent a plants structure but also its physical properties, with the dissadvantage that the L-system becomes more cumbersome and difficult to read. On this basis, it is beneficial to have a plant assets that not only are procedurally generated but also can be easily simulated. Further research is necessary to develop a straightforward method of writing the L-systems.   